\section{Theory and phenomenology of oscillations 15 pg SL}
\label{sec:th}


Neutrino oscillations are a quantum-mechanical phenomenon that is made possible
by the existence of non-degenerate neutrino masses and lepton flavour mixing.
As for quarks, the origin of flavour mixing in the lepton sector is the mismatch between
the basis of (weak) gauge eigenstates %(also called {\it flavour eigenstates})
and the basis of mass eigenstates,
namely the fact that the neutrino mass matrix is not diagonal when written in the flavour basis
(i.e. the weak eigenstate basis corresponding to the charged lepton mass eigenstates
$e$, $\mu$ and $\tau$).
%the neutrino to which a given charged lepton (electron, muon or tau) couples
%via the charged current is not a mass eigenstate, but a coherent superposition
%of mass eigenstates.
The relative rotation between the flavour and the mass eigenstate left-handed neutrino fields
is the {\it lepton mixing matrix}, known as the PMNS (Pontecorvo-Maki-Nakagawa-Sakata) matrix:
%
\be
  \left(\!\! \begin{array}{c} \nu_e(x) \\ \nu_\mu(x) \\ \nu_\tau(x) \end{array}\!\! \right)_{\!\!L}
  =\, U\, \left(\!\! \begin{array}{c} \nu_1(x) \\ \nu_2(x) \\ \nu_3(x) \end{array}\!\! \right)_{\!\!L}
  =\, \left(\! \begin{array}{ccc} U_{e1} & U_{e2} & U_{e3} \\ U_{\mu1} & U_{\mu2} & U_{\mu3} \\ U_{\tau 1}
    & U_{\tau2} & U_{\tau3} \end{array}\! \right)\! \left(\!\!
    \begin{array}{c} \nu_1(x) \\ \nu_2(x) \\ \nu_3(x) \end{array}\!\! \right)_{\!\!L}\, .
\label{eq:flavour_mass_relation_long}
\eeq
%
In Eq.~(\ref{eq:flavour_mass_relation_long}),
$\nu_{e L} (x)$, $\nu_{\mu L} (x)$ and $\nu_{\tau L} (x)$ are the fields describing
the left-handed {\it flavour eigenstate} neutrinos,
defined as the neutrinos that couple via charged weak current to the electron, the muon
and the tau, respectively,
and $\nu_{1 L} (x)$, $\nu_{2 L} (x)$ and $\nu_{3 L} (x)$ describe the left-handed
mass eigenstate neutrinos with masses $m_1$, $m_2$ and $m_3$.
%%
%\be
%  \left(\! \begin{array}{c} \nu_e(x) \\ \nu_\mu(x) \\ \nu_\tau(x) \end{array}\! \right)_{\!\!L}
%  =\, U_{\rm PMNS} \left(\! \begin{array}{c} \nu_1(x) \\ \nu_2(x) \\ \nu_3(x) \end{array}\! \right)_{\!\!L}
%  =\, \left(\! \begin{array}{ccc} U_{e1} & U_{e2} & U_{e3} \\ U_{\mu1} & U_{\mu2} & U_{\mu3} \\ U_{\tau 1}
%    & U_{\tau2} & U_{\tau3} \end{array}\! \right)\! \left(\!
%    \begin{array}{c} \nu_1(x) \\ \nu_2(x) \\ \nu_3(x) \end{array}\! \right)_{\!\!L}\, .
%\eeq
%%
In shorthand notations, the relation~(\ref{eq:flavour_mass_relation_long}) reduces to
%between the flavour and mass eigenstate neutrino fields
%
\be
  \nu_{\alpha L} (x)\, =\, \sum_i U_{\alpha i} \nu_{i L} (x)\, ,
\label{eq:flavour_mass_relation}
\eeq
%
where $\alpha = e, \mu, \tau$ and $i=1,2,3$.
In the following, we shall omit the subscript ``$L$'' for simplicity.
As a consequence of flavour mixing,
the neutrino that couples to a charged lepton of a given flavour (an electron,
a muon or a tau) is not a mass eigenstate, but a {\it coherent superposition of mass eigenstates}:
%%
%\be
%  {\cal L}_{\rm CC}\, =\, \frac{g}{\sqrt{2}}\ W^-_\mu \sum_\alpha\, \bar e_{\alpha L} \gamma^\mu \nu_{\alpha L}\,
%    =\, \frac{g}{\sqrt{2}}\ W^-_\mu \sum_{\alpha, i}\, U_{\alpha i}\, \bar e_{\alpha L} \gamma^\mu \nu_{i L}\, .
%\ee
%%
%
\be
  {\cal L}_{\rm CC}\, =\, \frac{g}{\sqrt{2}}\ W^-_\mu
    \sum_{\alpha = e, \mu, \tau}\! \bar e_{\alpha L} \gamma^\mu \nu_{\alpha L}\,
  =\, \frac{g}{\sqrt{2}}\ W^-_\mu \sum_{\alpha = e, \mu, \tau}\! \bar e_{\alpha L} \gamma^\mu
    \sum_{i = 1,2,3}\! U_{\alpha i}\, \nu_{i L}\, .
\label{eq:L_CC}
\eeq
%
%%
%\be
%  {\cal L}_{\rm CC}\, =\, \frac{g}{\sqrt{2}}\ W^-_\mu\!
%    \sum_{\alpha = e, \mu, \tau}\! \bar e_{\alpha L} \gamma^\mu \nu_{\alpha L}\,
%  =\, \frac{g}{\sqrt{2}}\ W^-_\mu\! \sum_{\alpha = e, \mu, \tau}\! \bar e_{\alpha L} \gamma^\mu\,
%    \sum_i\, U_{\alpha i} \nu_{i L}\, .
%\label{eq:L_CC}
%\eeq
%%
%%
%\be
%  {\cal L}_{\rm CC}\, =\, \frac{g}{\sqrt{2}}\ W^-_\mu \sum_\alpha\, \bar e_{\alpha L} \gamma^\mu \nu_{\alpha L}\,
%    =\, \frac{g}{\sqrt{2}}\ W^-_\mu \sum_\alpha\, \bar e_{\alpha L} \gamma^\mu \sum_i\, U_{\alpha i} \nu_{i L}\, .
%\eeq
%%
It is this coherence that makes neutrino oscillations possible.
%\omi{Neutral current, on the contrary,
%is not sensitive to flavour mixing, since $\sum_\alpha \bar \nu_{\alpha L} \gamma^\mu \nu_{\alpha L}
%= \sum_{i,j} \sum_\alpha U^*_{\alpha j} U_{\alpha i}\, \bar \nu_{j L} \gamma^\mu \nu_{i L}
%= \sum_{i,j} \delta_{i j}\, \bar \nu_{j L} \gamma^\mu \nu_{i L} = \sum_i \bar \nu_{i L} \gamma^\mu \nu_{i L}$.}

Being associated with a change of basis, the PMNS matrix is a unitary matrix. Like the CKM matrix,
it satisfies unitary relations, derived from $U U^\dagger = U^\dagger U = \mathbf{1}$:
%
\be
  \sum_i\, U_{\alpha i} U^*_{\beta i}\, =\, \delta_{\alpha \beta} \quad \left( \alpha, \beta = e, \mu, \tau \right) ,
  \qquad  \sum_\alpha\, U^*_{\alpha i} U_{\alpha j}\, =\, \delta_{ij} \quad \left( i, j = 1,2,3 \right) .
\eeq
%
Like any $3 \times 3$ unitary matrix, $U$ can be parametrized by 3 mixing angles and 6 phases.
However, not all of these phases are physical, since lepton fields can be rephased
to absorb some of them. Namely, if neutrinos are Dirac fermions, one can rephase
both the charged lepton and the neutrino fields, $e_\alpha (x) \to e^{i\phi_\alpha}\, e_\alpha (x)$
and $\nu_i (x) \to e^{i\phi_i}\, \nu_i (x)$, where $e_\alpha (x)$ and $\nu_i (x)$ denote
the 4-component Dirac fields (i.e. the phases of the left-handed and right-handed lepton fields
are shifted by the same amount, in order not to affect the mass terms
$- \sum_\alpha m_{\ell_\alpha} \bar e_{R \alpha} e_{L \alpha} - \sum_i m_i  \bar \nu_{R i} \nu_{L i} + \mbox{h.c.}$).
%This leaves the Lagrangian invariant, including the charged current
%term~(\ref{eq:L_CC}), provided that the PMNS matrix is redefined in the following way:
This leaves the charged current term~(\ref{eq:L_CC}) invariant, provided that one redefines
the PMNS matrix in the following way:
%
\be
  U_{\alpha i}\, \to\, e^{i (\phi_\alpha - \phi_i)}\, U_{\alpha i}\, .
\eeq
%
Since there are 5 independent phase differences $\phi_\alpha - \phi_i$, one can
remove 5 phases from the PMNS matrix, leaving only one physical CP-violating
phase, as in the CKM matrix. If neutrinos are Majorana fermions, however, it is
not possible to rephase the left-handed neutrino fields, because this would make
their masses complex (indeed, Majorana mass terms are of the form
$- \frac{1}{2}\, m_i\, \nu^T_{L i} C \nu_{L i} + \mbox{h.c.}$, where $C$ is the charge
conjugation matrix satisfying $C \gamma_\mu C^{-1} = - \gamma^T_\mu$).
%Thus only the charged lepton fields can be rephased and only 3 phases can be removed
%from the PMNS matrix (as $U_{\alpha i}\, \to\, e^{i \phi_\alpha}\, U_{\alpha i}$),
%leaving 3 physical CP-violating phases.
Thus only the charged lepton fields can be rephased, leading to
%
\be
  U_{\alpha i}\, \to\, e^{i \phi_\alpha}\, U_{\alpha i}\, .
\eeq
%
One is therefore left with 3 physical CP-violating phases in the Majorana case,
instead of a single one in the Dirac case\footnote{This parameter counting can be
generalized to an arbitrary number $N$ of lepton flavours,
%but this exercise is of rather academic interest, given that the number of active neutrinos
%is known to be 3 without ambiguity (sterile neutrinos may however exist, but the counting
although the number of active neutrinos
is known to be 3 without ambiguity (sterile neutrinos may exist, but the counting
of physical mixing parameters is different in this case, see Section~\ref{subsec:steriles}).
One finds, in the Dirac case, $N(N-1)/2$ mixing angles
and $(N-1)(N-2)/2$ phases, and $N-1$ additional phases in the Majorana case.
Thus, at variance with the quark sector, CP violation is possible already with 2 generations
of leptons if neutrinos are Majorana fermions. Nevertheless, for reasons that will become clear below,
CP violation in oscillations requires at least 3 generations.}.


%The standard parametrization of the PMNS matrix is
%The PMNS matrix can therefore be written as the product of three rotations
Based on this parameter counting, the PMNS matrix can be written as the product of three rotations
through angles $\theta_{23}$, $\theta_{13}$ and $\theta_{12}$,  where the second
(unitary) rotation depends on a phase $\delta$, and of a diagonal matrix of phases $P$:
%
\bea
  U\! & =\! & \left( \begin{array}{ccc} 1 & 0 & 0 \\ 0 & c_{23} & s_{23} \\ 0 & - s_{23} & c_{23} \end{array} \right)
    \left( \begin{array}{ccc} c_{13} & 0 & s_{13} e^{-i \delta} \\ 0 & 1 & 0 \\ - s_{13} e^{i \delta} & 0 & c_{13} \end{array} \right)
    \left( \begin{array}{ccc} c_{12} & s_{12} & 0 \\ - s_{12} & c_{12} & 0 \\ 0 & 0 & 1 \end{array} \right) P  \nn \\
  & =\! & \left( \begin{array}{ccc} c_{12} c_{13} & s_{12} c_{13} & s_{13} e^{-i \delta} \\
    - s_{12} c_{23} - c_{12} s_{13} s_{23} e^{i \delta} & c_{12} c_{23} - s_{12} s_{13} s_{23} e^{i \delta} & c_{13} s_{23} \\
    s_{12} s_{23} - c_{12} s_{13} c_{23} e^{i \delta} & - c_{12} s_{23} - s_{12} s_{13} c_{23} e^{i \delta} & c_{13} c_{23} 
   \end{array} \right) P\ .
\label{eq:PMNS}
\eea
%
In Eq.~(\ref{eq:PMNS}), $c_{ij} \equiv \cos \theta_{ij}$, $s_{ij} \equiv \sin \theta_{ij}$ and $P$
is either the unit matrix $\mathbf{1}$ in the Dirac case, or a diagonal matrix containing
the two phases associated with the Majorana nature of neutrinos in the Majorana case.
Without loss of generality, one can take $\theta_{ij} \in \left[ 0, \frac{\pi}{2} \right]$ and
$\delta \in \left[ 0, 2 \pi \right[$.
%Except for the explicit form of $P$ in the Majorana case, this parametrization has now become
%standard. The following conventions for the Majorana phases can be found in the literature:
This parametrization has now become standard, except for the explicit form of the matrix $P$
in the Majorana case, for which different conventions can be found in the literature, e.g.
%
\be
  P_{\rm Majorana}\, =\, \left( \begin{array}{ccc} e^{i \alpha_1} & 0 & 0 \\ 0 & e^{i \alpha_2} & 0 \\ 0 & 0 & 1 \end{array} \right) ,
    \quad %\mbox{or} \quad
  \left( \begin{array}{ccc} 1 & 0 & 0 \\ 0 & e^{i \phi_2} & 0 \\ 0 & 0 & e^{i \phi_3} \end{array} \right) ,
    \quad %\mbox{or} \quad
  \left( \begin{array}{ccc} e^{i \rho} & 0 & 0 \\ 0 & 1 & 0 \\ 0 & 0 & e^{i \sigma} \end{array} \right) .
    %\quad \mbox{or \dots}  \hskip .3cm
\eeq
%
All these choices are related by rephasings of the charged lepton fields and are
therefore physically equivalent.
%All these conventions are physically equivalent, since one can change from one to another
%by an overall rephasing of the charged lepton fields, $e_\alpha (x) \to e^{i\phi}\, e_\alpha (x)$.
The phase $\delta$ is often called the ``Dirac phase'' of the PMNS matrix, while
the phases contained in $P$, which can be restricted to the range $\left[ 0, \pi \right[$
without loss of generality, are dubbed ``Majorana phases''. 
We will not be concerned with Majorana phases in this review because, as we are
going to see, they do not enter oscillation probabilities.
They appear only in lepton number violating processes like neutrinoless double beta decay,
in which the Majorana nature of neutrinos plays a crucial role.
The phase $\delta$, on the contrary, is relevant to neutrino oscillations and gives
rise to an asymmetry between neutrino and antineutrino oscillations in vacuum,
as will be discussed in Section~\ref{subsec:CPV}.

%The parameter counting performed in the three-generation case can be
%generalized to an arbitrary number $N$ of lepton flavours,
%although the number of active neutrinos
%is known to be 3 without ambiguity (sterile neutrinos may exist, but the counting
%of physical mixing parameters is different in this case, see Section~\ref{subsec:steriles}).
%One finds, in the Dirac case, $N(N-1)/2$ mixing angles
%and $(N-1)(N-2)/2$ phases, and $N-1$ additional phases in the Majorana case.
%Thus, at variance with the quark sector, CP violation is possible already with 2 generations
%of leptons if neutrinos are Majorana fermions. Nevertheless, CP violation in oscillations
%requires at least 3 generations.


%%%%%%%%%%%%%%%%%%%%%%
\subsection{\it Oscillations in vacuum     %
\label{subsec:vacuum}}                          %
%%%%%%%%%%%%%%%%%%%%%%

Schematically, an (idealized) oscillation experiment involves three steps.
The first one is the production of a pure flavour state from a charged current process,
e.g. a $\nu_\mu$ from a charged pion decay $\pi^+ \to \mu^+ \nu_\mu$. This flavour eigenstate
is a coherent superposition of mass eigenstates determined by the PMNS matrix\footnote{Note
that the relation between the flavour and mass eigenstate neutrino {\it states} involves the
complex conjugate of the PMNS matrix, as opposed to the PMNS matrix itself for neutrino
{\it fields}, Eq.~(\ref{eq:flavour_mass_relation}). This is because the quantum neutrino field
$\nu_\alpha(x)$ annihilates a neutrino of flavour $\alpha$, while the neutrino state
$\left| \nu_\alpha (\vec{p}) \right>$ is obtained by acting with the creation operator
$a^\dagger_\alpha (\vec{p})$ on the vacuum. For antineutrinos, one has
$\bar \nu_\alpha(x) = \sum_i U^*_{\alpha i} \bar \nu_i(x)$ and
$|\bar \nu_\alpha \rangle = \sum_i U_{\alpha i} |\bar \nu_i\rangle$.}:
%
\be
  |\nu (t=0) \rangle\, =\, |\nu_\alpha \rangle\, =\, \sum_i U^*_{\alpha i} |\nu_i\rangle\, .
\eeq
%
The second step is the propagation of the neutrino. Each mass eigenstate,
being en eigenstate of the Hamiltonian in vacuum, evolves with its own phase factor
$e^{-i E_i t}$ (in the standard convention $\hbar = 1$),
where $E_i = \sqrt{p^2 + m^2_i}$ is the energy of the $i$-th mass eigenstate.
This modifies the coherent superposition, which is no longer a pure flavour eigenstate:
%
\be
  |\nu (t) \rangle\, =\, \sum_i U^*_{\alpha i}\, e^{-i E_i t}\, |\nu_i \rangle\,
    =\, \sum_i U^*_{\alpha i}\, e^{-i E_i t} \sum_\beta U_{\beta i} |\nu_\beta \rangle\, .
\eeq
%
The last step is the detection of a specific flavour via a charged current interaction.
The probability amplitude for the flavour $\alpha$ neutrino to have oscillated into
a different flavour $\beta$ at the time $t$ is given by $\langle \nu_\beta |\nu (t) \rangle$,
yielding an oscillation probability
%
\be
  P (\nu_\alpha \to \nu_\beta)\, =\, \left| \langle \nu_\beta |\nu (t) \rangle \right|^2\,
    =\, \left| \sum_i U_{\beta i} U^*_{\alpha i}\, e^{-i E_i t} \right|^2\, .
\eeq
%
Since neutrinos are ultra-relativistic in all practical experimental conditions, one
can expand $E_i = \sqrt{p^2 + m^2_i}\, \simeq\, p + m^2_i/(2E)$ (using $E \simeq p$).
The neutrino oscillation formula then reads\footnote{In deriving the oscillation
formula~(\ref{eq:oscillation_probability}), we made several simplifying assumptions:
the propagating mass eigenstates were described by plane waves with well-defined
momenta $\vec{p}_i$, which were further assumed to be equal ($\vec{p}_i = \vec{p}$).
The proper treatment of neutrino oscillations should be done in the wave-packet
formalism, or even in the framework of quantum field theory. However, under appropriate
coherence conditions, these approaches lead to the same oscillation probability
as the standard derivation presented here. See e.g. [Akhmedov, Smirnov] for a discussion.}
%
\bea
  P (\nu_\alpha \to \nu_\beta)\ = &\! \delta_{\alpha \beta}\!\! &
    -\ 4 \sum_{i < j}\, \mbox{Re} \left[ U_{\alpha i} U^*_{\beta i} U^*_{\alpha j} U_{\beta j} \right]
    \sin^2 \left( \frac{\Delta m^2_{ji} L}{4 E} \right)  \nn \\
    && +\ 2 \sum_{i < j}\, \mbox{Im} \left[ U_{\alpha i} U^*_{\beta i} U^*_{\alpha j} U_{\beta j} \right]
    \sin \left( \frac{\Delta m^2_{ji} L}{2 E} \right) ,
\label{eq:oscillation_probability}
\eea
%
in which $\Delta m^2_{ji} \equiv m^2_j - m^2_i$ and $L \simeq c t$ is the distance
travelled by the neutrino. For antineutrino oscillations, one must replace $U$ by
$U^*$ in Eq.~(\ref{eq:oscillation_probability}), which amounts to change the sign
of the last term.

From Eq.~(\ref{eq:oscillation_probability}), a few obvious comments can be made
about the properties of neutrino oscillations. First of all, oscillations require neutrinos
to have non-degenerate masses ($\Delta m^2_{ji} \neq 0$) and non-trivial flavour
mixing ($U \neq \mathbf{1}$). The oscillation probability $P (\nu_\alpha \to \nu_\beta)$
depends on the three mixing angles $\theta_{12}$, $\theta_{23}$, $\theta_{13}$
and on two independent squared-mass differences, which can be chosen to be
$\Delta m^2_{21}$ and $\Delta m^2_{31}$ (then $\Delta m^2_{32}$ is determined
by $\Delta m^2_{32} = \Delta m^2_{31} - \Delta m^2_{21}$).
Oscillations also depend on the ``Dirac'' CP-violating phase $\delta$, but not
on the ``Majorana phases'', as can be seen from the fact that the PMNS matrix
entries appear in Eq.~(\ref{eq:oscillation_probability}) only
in the combinations $U_{\alpha i} U^*_{\beta i}$, to which the phases contained
in the diagonal matrix $P$ in Eq.~(\ref{eq:PMNS}) do not contribute.
Therefore, Dirac and Majorana
neutrinos have the same oscillation probabilities. This fact can be understood
on more general grounds, since oscillations %violate lepton flavour but
conserve total lepton number,
while the Majorana nature of neutrinos manifests itself in processes that violate
lepton number, like neutrinoless double beta decay.
%
Another consequence of formula~(\ref{eq:oscillation_probability}) is that
CP violation (namely, the fact that
$P (\bar \nu_\alpha \to \bar \nu_\beta) \neq P (\nu_\alpha \to \nu_\beta)$)
is possible only in appearance channels ($\beta \neq \alpha$), not in disappearance
channels ($\beta = \alpha$). Indeed, the survival (i.e. non-oscillation) probability
is the same for neutrinos and antineutrinos:
%
\bea
  P (\nu_\alpha \to \nu_\alpha)\, =\, 1 - 4 \sum_{i < j}\, \left| U_{\alpha i} U_{\alpha j} \right|^2
    \sin^2 \left( \frac{\Delta m^2_{ji} L}{4 E} \right) =\, P (\bar \nu_\alpha \to \bar \nu_\alpha) \, ,
\eea
%
because the combination $U_{\alpha i} U^*_{\beta i} U^*_{\alpha j} U_{\beta j}$
is real for $\alpha = \beta$.

While a detailed description of neutrino oscillations, including subleading and
CP-violation effects, necessitates the use of the full three-flavour formula,
it is a good aproximation in some experimental situations to neglect the subleading
terms in Eq.~(\ref{eq:oscillation_probability}). One is then left with effective
two-flavour oscillations, governed by a single $\Delta m^2$ and a single mixing
angle $\theta$:
%%
%\be
%  P (\nu_\alpha \to \nu_\beta)\, =\, \left\{ \begin{array}{cl}
%    \sin^2 2 \theta\, \sin^2 \left( \frac{\Delta m^2 L}{4E} \right) & \mbox{appearance mode}\ (\beta \neq \alpha)  \\
%    1 - \sin^2 2 \theta\, \sin^2 \left( \frac{\Delta m^2 L}{4E} \right) & \mbox{disappearance mode}\ (\beta = \alpha)
%    \end{array} \right.\ .
%\eeq
%%
%
\be
  P (\nu_\alpha \to \nu_\beta)\, =\, P (\bar \nu_\alpha \to \bar \nu_\beta)\, =\,
    \sin^2 2 \theta\, \sin^2 \left( \frac{\Delta m^2 L}{4E} \right) \qquad (\beta \neq \alpha)\, ,
\label{eq:2f_oscillations}
\eeq
%
(we note in passing that oscillations do not violate the  CP symmetry in the two-flavour case).
%as the probabilities are the same for neutrinos and antineutrinos).
In this case, the amplitude of oscillations is simply given by $\sin^2 2 \theta$,
and the oscillation length is proportional to the neutrino energy $E$ and inversely
proportional to $\Delta m^2$:
%
\be
  L_{\rm osc.} \mbox{(km)}\, =\, 2.48\, E \mbox{(GeV)} / \Delta m^2 (\mbox{eV}^2)\, , \qquad
  P (\nu_\alpha \to \nu_\beta)\, =\, \sin^2 2 \theta\, \sin^2 \left( \pi L / L_{\rm osc.} \right) .
\eeq
%
%Another useful formula is
%%
%\be
%  P (\nu_\alpha \to \nu_\beta)\, =\, \sin^2 2 \theta\,
%    \sin^2 \left( 1.27\ \frac{\Delta m^2 (\mbox{eV}^2) L (\mbox{km})}{4E (\mbox{GeV})} \right) .
%\eeq
%%
This is the well-known two-flavour oscillation formula, which has been used
in phenomenological studies for decades, before experiments reached
a level of precision that made them sensitive to subdominant effects.
In particular, the oscillations of solar and atmospheric neutrinos turned out
to be well described by the two-flavour formula~(\ref{eq:2f_oscillations}) with parameters
($\Delta m^2_\odot$, $\theta_\odot$) and ($\Delta m^2_\oplus$, $\theta_\oplus$),
respectively, such that $\Delta m^2_\odot \ll \Delta m^2_\oplus$ and both mixing
angles $\theta_\odot$ and $\theta_\oplus$ are large. In order to interpret these
results in the framework of three-flavour oscillations, the following conventions
were adopted: {\it (i)} $\Delta m^2_\odot$ is identified with the squared-mass splitting
between $\nu_1$ and $\nu_2$; {\it (ii)} these mass eigenstates are labelled
in such a way that $m_2 > m_1$, i.e. $\Delta m^2_{21} = \Delta m^2_\odot > 0$.
%{\it (i)} the smaller squared-mass difference $\Delta m^2_\odot$
%is identified with $|\Delta m^2_{21}|$; {\it (ii)} the corresponding mass eigenstates
%$\nu_1$ and $\nu_2$ are labelled in such a way that $m_2 > m_1$,
%i.e. $\Delta m^2_{21} = \Delta m^2_\odot > 0$.
%Once this is done, there is no other choice than identifying $\Delta m^2_\oplus$
Then $\Delta m^2_\oplus$ must be identified with $|\Delta m^2_{31}|$ or $|\Delta m^2_{32}|$.
Since $\Delta m^2_\odot \ll \Delta m^2_\oplus$, this implies
%
\be
  \Delta m^2_\odot = \Delta m^2_{21}\, \ll\, |\Delta m^2_{31}| \simeq |\Delta m^2_{32}| \simeq \Delta m^2_\oplus\, .
\label{eq:Delta_m2}
\eeq
%
%%
%\be
%  \Delta m^2_{21}\, \ll\, |\Delta m^2_{31}| \simeq |\Delta m^2_{32}|\, .
%\label{eq:Delta_m2}
%\eeq
%%
We are left with two possibilities for the mass spectrum: either $m_1 < m_2 < m_3$,
which is referred to as the normal mass ordering or {\it normal hierarchy}, characterized by
$\Delta m^2_{31} > 0$; or $m_3 < m_1 < m_2$, which is known as the inverted mass ordering
or {\it inverted hierarchy}, characterized by $\Delta m^2_{31} < 0$. The mass ordering
$m_1 < m_3 < m_2$ is excluded, as it is not consistent with Eq.~(\ref{eq:Delta_m2}).
%not allowed, as it would imply $|\Delta m^2_{31}| < \Delta m^2_{21}$.
%\com{faut-il ajouter un sch\'ema?}

Experiments characterized by a baseline $L$ and a beam energy $E$ such that $\Delta m^2_{21} L / E \ll 1$
can be described to a good approximation by setting $\Delta m^2_{21} = 0$ in the
three-flavour formula~(\ref{eq:oscillation_probability}), so that one is left with a single
oscillation ``frequency'' $\Delta m^2_{31} = \Delta m^2_{32}$. The oscillation (appearance)
probability becomes
%
\be
  P (\nu_\alpha \to \nu_\beta)\, =\, \sin^2 2 \theta^{\rm eff}_{\alpha \beta}\,
    \sin^2 \left( \frac{\Delta m^2_{31} L}{4E} \right) ,  \qquad
    \sin^2 2 \theta^{\rm eff}_{\alpha \beta}\, \equiv\, 4 \left| U_{\alpha 3} U_{\beta 3} \right|^2 \qquad (\beta \neq \alpha)\, ,
\eeq
%
while for the non-oscillation (disappearance) probability:
%
\be
  P (\nu_\alpha \to \nu_\alpha)\, =\, 1 - \sin^2 2 \theta^{\rm eff}_{\alpha \alpha}\,
    \sin^2 \left( \frac{\Delta m^2_{31} L}{4E} \right) ,  \qquad  \sin^2 2 \theta^{\rm eff}_{\alpha \alpha}\,
    \equiv\, 4 \left| U_{\alpha 3} \right|^2 \left( 1 - \left| U_{\alpha 3} \right|^2 \right) .
\eeq
%
These formulae describe the dominant oscillations in atmospheric neutrinos,
long-baseline accelerator neutrino experiments and short-baseline reactor experiments.
For instance, the probability of muon neutrino disappearance is
%
\bea
  P (\nu_\mu \to \nu_\mu)\!\! & =\!\! &  1 - 4 \cos^2 \theta_{13} \sin^2 \theta_{23}\!
    \left( 1 - \cos^2 \theta_{13} \sin^2 \theta_{23} \right) \sin^2 \left( \frac{\Delta m^2_{31} L}{4E} \right)  \\
  & \simeq\!\! & 1 - \sin^2 2 \theta_{23}\, \sin^2 \left( \frac{\Delta m^2_{31} L}{4E} \right) ,
\eea
%
where terms proportional to $\sin^2 \theta_{13}$ were neglected in the second line.
This justifies the widely-used terminology ``atmospheric mixing angle'' for $\theta_{23}$
and ``atmospheric $\Delta m^2$'' for $\Delta m^2_{31}$ (or $\Delta m^2_{32}$).
For electron and tau neutrino appearance in a muon neutrino beam, one has
%
\bea
  P (\nu_\mu \to \nu_e)\!\! & =\!\! & \sin^2 \theta_{23} \sin^2 2 \theta_{13}\, \sin^2 \left( \frac{\Delta m^2_{31} L}{4E} \right) ,  \\
  P (\nu_\mu \to \nu_\tau)\!\! & =\!\! & \cos^4 \theta_{13} \sin^2 2 \theta_{23}\, \sin^2 \left( \frac{\Delta m^2_{31} L}{4E} \right) ,
\eea
%
while for short-baseline disappearance of reactor antineutrinos:
%
\be
  P (\bar \nu_e \to \bar \nu_e)\, =\, 1 - \sin^2 2 \theta_{13}\, \sin^2 \left( \frac{\Delta m^2_{31} L}{4E} \right) .
\eeq
%
It should be kept in mind that the above expressions receive corrections from three-flavour effects,
which must be taken into account to comply with the precision of present-day experiments.

%When instead $\Delta m^2_{31} L / E \gg 1$ and $\Delta m^2_{21} L / E \gtrsim 1$,
%$\Delta m^2_{31}$-driven oscillations are averaged and oscillations of electron neutrinos
%are dominated by $\Delta m^2_{21}$ rather than $\Delta m^2_{31}$.
%%
%\bea
%  P (\nu_e \to \nu_e)\, =\, P (\bar \nu_e \to \bar \nu_e) & = & \sin^4 \theta_{13}
%    + \cos^4 \theta_{13} \left( 1 - \sin^2 2 \theta_{12}\, \sin^2 \left( \frac{\Delta m^2_{21} L}{4E} \right) \right)  \\
%  & \simeq & 1 - \sin^2 2 \theta_{12}\, \sin^2 \left( \frac{\Delta m^2_{21} L}{4E} \right) ,
%\eea
%%
%where again terms suppressed by $\sin^2 \theta_{13}$ were neglected in the second line.

When instead $\Delta m^2_{31} L / E \gg 1$ and $\Delta m^2_{21} L / E \gtrsim 1$,
$\Delta m^2_{31}$-driven oscillations are averaged and oscillations of electron neutrinos
are dominated by $\Delta m^2_{21}$ rather than $\Delta m^2_{31}$.
Neglecting terms suppressed by $\sin^2 \theta_{13}$, one obtains
%
\be
  P (\nu_e \to \nu_e)\, =\, P (\bar \nu_e \to \bar \nu_e)\, \simeq\,
    1 - \sin^2 2 \theta_{12}\, \sin^2 \left( \frac{\Delta m^2_{21} L}{4E} \right) .
\eeq
%
This formula applies to the long-baseline reactor neutrino experiment KamLAND and to
low-energy solar neutrinos (with the oscillating term averaged), for which matter effects
are subdominant compared with vacuum oscillations. This justifies the popular terminology
``solar mixing angle'' for $\theta_{12}$ and ``solar $\Delta m^2$'' for $\Delta m^2_{21}$.
Restoring $\theta_{13}$, one obtains the more accurate expression
%
\be
  P (\nu_e \to \nu_e)\, =\, P (\bar \nu_e \to \bar \nu_e)\, =\, \sin^4 \theta_{13}
    + \cos^4 \theta_{13} \left( 1 - \sin^2 2 \theta_{12}\, \sin^2 \left( \frac{\Delta m^2_{21} L}{4E} \right) \right) .
\label{eq:nue_nue_2f_improved}
\eeq
%


%%%%%%%%%%%%%%%%%%%%%%%%%%%%%%%%%%%%%
\subsection{\it CP violation in oscillations and three-flavour effects    %
\label{subsec:CPV}}                                                                         %
%%%%%%%%%%%%%%%%%%%%%%%%%%%%%%%%%%%%%

As we have seen in Section~\ref{subsec:vacuum}, oscillations depend on the
phase $\delta$ of the PMNS matrix, making it possible to observe the violation of the CP
symmetry in neutrino oscillations -- namely, the fact that neutrinos and antineutrinos
oscillate with different probabilities. Before discussing this possibility in detail, let us
summarize the action of the different discrete symmetries on oscillation probabilities\footnote{It
should be stressed that $P (\bar \nu_\alpha \to \bar \nu_\beta)$ is the image of
$P (\nu_\alpha \to \nu_\beta)$ by CP, not by the charge conjugation C. Indeed,
$\nu_\alpha$ and $\nu_\beta$ are left-handed neutrinos, whose antiparticles
$\bar \nu_\alpha$ and $\bar \nu_\beta$ are right-handed antineutrinos, i.e. the
CP conjugates of $\nu_\alpha$ and $\nu_\beta$. Their charge conjugates would be
hypothetical left-handed antineutrinos, which do not couple to the $W$ and $Z$ bosons
and are not produced in weak processes.}:
%
\bea
  P (\nu_\alpha \to \nu_\beta) & \xrightarrow{\mbox{  CP  }} & P (\bar \nu_\alpha \to \bar \nu_\beta)\ ,  \\
  & \xrightarrow{\mbox{\phantom{P}T\phantom{P}}} & P (\nu_\beta \to \nu_\alpha)\ ,  \\
  & \xrightarrow{\mbox{CPT}} & P (\bar \nu_\beta \to \bar \nu_\alpha)\ .
\eea
%
Thus, if CPT conservation is assumed, $P (\nu_\alpha \to \nu_\beta) = P (\bar \nu_\beta \to \bar \nu_\alpha)$,
implying that the CP and T asymmetries in neutrino oscillations are equal:
%
\be
  A_{\alpha \beta}\, \equiv\, \frac{P (\nu_\alpha \to \nu_\beta) - P (\bar \nu_\alpha \to \bar \nu_\beta)}
    {P (\nu_\alpha \to \nu_\beta) + P (\bar \nu_\alpha \to \bar \nu_\beta)}\,
  =\, \frac{P (\nu_\alpha \to \nu_\beta) - P (\nu_\beta \to \nu_\alpha)}
    {P (\nu_\alpha \to \nu_\beta) + P (\nu_\beta \to \nu_\alpha)}\ .
\eeq
%
Another implication of CPT conservation is $P (\nu_\alpha \to \nu_\alpha) = P (\bar \nu_\alpha \to \bar \nu_\alpha)$,
i.e. there is no CP violation in disappearance experiments, a fact we have
already deduced from the general three-flavour oscillation formula.

In order to study the effect of CP violation in neutrino oscillations, it is convenient
to introduce the quantity
$\Delta P_{\alpha \beta} \equiv P (\nu_\alpha \to \nu_\beta) - P (\bar \nu_\alpha \to \bar \nu_\beta)$,
which is nothing but twice the CP-odd part in the three-flavour oscillation
probability~(\ref{eq:oscillation_probability}). One can show that it can be expressed as
%
\be
  \Delta P_{\alpha \beta}\, =\, \pm 16 J\, \sin \left( \frac{\Delta m^2_{21} L}{4 E} \right)
    \sin \left( \frac{\Delta m^2_{31} L}{4 E} \right) \sin \left( \frac{\Delta m^2_{32} L}{4 E} \right) ,\ \ \quad
    J\, \equiv\, \mbox{Im} \left[ U_{e 1} U^*_{\mu 1} U^*_{e 2} U_{\mu 2} \right] ,
\label{eq:CP_asymmety}
\eeq
%
%which is a rewriting of
%%
%\be
%  \Delta P_{\alpha \beta}\, =\, \pm 4\, J
%    \left( \sin \left( \frac{\Delta m^2_{21} L}{2 E} \right) - \sin \left( \frac{\Delta m^2_{31} L}{2 E} \right)
%    + \sin \left( \frac{\Delta m^2_{32} L}{2 E} \right) \right) .
%\eeq
%%
with a $+$ sign when $(\alpha, \beta, \gamma)$ (with $\gamma \neq \alpha, \beta$)
is an even permutation of $(e, \mu, \tau)$, and a $-$ sign when it is an odd permutation.
The quantity $J$ in Eq.~(\ref{eq:CP_asymmety}) is a measure of CP violation from the
``Dirac'' phase of the PMNS matrix and is called {\it Jarlskog invariant}~\!\footnote{The name
``invariant'' refers to the fact that $J$ does not depend on the phase convention of the PMNS
matrix, i.e. it is invariant under rephasings of the lepton fields. Historically, the Jarlskog
invariant has been introduced to parametrize CP violation in the quark sector;
%There it can be expressed as $J = \mbox{Im} \left[ V_{ud} V^*_{c d} U^*_{us} V_{cs} \right]$,
%where $V$ denotes the CKM matrix.
the invariant $J$ in Eq.~(\ref{eq:CP_asymmety}) is its generalization to the lepton sector.}.
Using the standard parametrization of the PMNS matrix, one can write
%
\be
  J\, =\, \frac{1}{8}\, \cos \theta_{13} \sin 2 \theta_{12} \sin 2 \theta_{13} \sin 2 \theta_{23} \sin \delta\, .
\eeq
%
Therefore, a necessary condition for CP violation in neutrino oscillations
is that all three mixing angles $\theta_{ij}$ are nonzero and that the phase
$\delta$ is different from $0$ and $\pi$.
Furthermore, one can see from Eq.~(\ref{eq:CP_asymmety}) that $\Delta m^2_{21}$,
$\Delta m^2_{31}$ and $\Delta m^2_{32}$ must be non vanishing, i.e. all neutrinos masses
should be different:
%
\bea
  \mbox{\it Conditions for CPV in oscillations :} \qquad
  \theta_{ij} \neq 0\, , \quad \delta \neq 0, \pi\, , \quad m_1 \neq m_2\, , \ m_2 \neq m_3\, , \ m_3 \neq m_1\, .
\eea
%
These criteria parallel the ones for CP violation in the quark sector.
They are of rather academic interest now that all $\theta_{ij}$'s and
$\Delta m^2_{ji}$'s have been measured, but until 2011 the only
experimental information we had on $\theta_{13}$ was an upper bound.
A value of $\theta_{13}$ below $1^\circ$ would have made it extremely
difficult to observe CP violation in oscillation experiments, even if the CP-violating
phase $\delta$ were maximal.

The formula~(\ref{eq:CP_asymmety}) contains a lot of information. First of all, it tells us
that the CP-violating term in neutrino oscillations is universal, i.e. it does not depend
on the oscillation channel (up to a sign). This follows from the unitarity of the PMNS
%matrix ($\sum_\alpha U_{\alpha 1} U^*_{\alpha 2} = 0$), which implies
matrix, which implies
%
\bea
  \mbox{Im} \left[ U_{e 1} U^*_{\mu 1} U^*_{e 2} U_{\mu 2} \right]
    = - \mbox{Im} \left[ U_{e 1} U^*_{\tau 1} U^*_{e 2} U_{\tau 2} \right]
    = \mbox{Im} \left[ U_{\mu 1} U^*_{\tau 1} U^*_{\mu 2} U_{\tau 2} \right] .
\eea
%
Another important information contained in Eq.~(\ref{eq:CP_asymmety}) is that
the effect of CP violation is proportional to $\sin ( \Delta m^2_{21} L / 4 E )$,
i.e. it can be observed only in experiments that are sensitive to the subdominant
oscillations governed by $\Delta m^2_{21}$. This is why experiments searching
for CP violation involve long baselines (several hundreds of km),
intense neutrino beams and large detectors.
%(e.g. a 50 kton liquid scintillator detector, a 100 kton liquid Argon TPC
%or a Mton-class water \v{C}erenkov detector).
Typically, the experimental conditions are such that $\Delta m^2_{31} L / E \sim 1$
and $\Delta m^2_{21} L / E \ll 1$, hence Eq.~(\ref{eq:CP_asymmety}) can be simplified to,
at second order in the small parameters $\sin \left( \Delta m^2_{21} L / 4 E \right)$ and $\sin \theta_{13}$:
%at second order in the small parameters $\Delta m^2_{21} / \Delta m^2_{31}$ and $\theta_{13}$:
%%
%\be
%  \Delta P_{\alpha \beta}\, \simeq\, \pm 16 J
%    \left( \frac{\Delta m^2_{21} L}{4 E} \right) \sin^2 \left( \frac{\Delta m^2_{31} L}{4 E} \right) .
%\label{eq:CP_asymmety_approx}
%\eeq
%%
%
\be
  \Delta P_{\alpha \beta}\, \simeq\, \pm 16 J\,
    \sin \left( \frac{\Delta m^2_{21} L}{4 E} \right) \sin^2 \left( \frac{\Delta m^2_{31} L}{4 E} \right) .
\label{eq:CP_asymmety_approx}
\eeq
%
Long baselines however imply that neutrinos propagate in the Earth crust, so that
their oscillations are affected by their interactions with matter (see Section~\ref{subsec:constant}).
This in turn creates, in the $\nu_\mu$--$\nu_e$ channel relevant to long-baseline experiments,
an asymmetry between neutrino and antineutrino oscillations
whose sign is related to the type of mass hierarchy,
normal ($\Delta m^2_{31} > 0$) or inverted ($\Delta m^2_{31} < 0$).
%It is therefore necessary to disentangle matter effects from CP violation
%in the experimental data.
It is therefore necessary to disentangle the effect of neutrino interactions with matter
from the one of CP violation in the experimental data.
%from the one of the CP-violating phase $\delta$ in the experimental data.

Neglecting matter effects (which is a reasonable approximation for a long-baseline experiment
like T2K), one can expand the full $\nu_\mu \to \nu_e$ oscillation probability to second order
%in the small quantities $\Delta m^2_{21} / \Delta m^2_{31}$ and $\sin \theta_{13}$:
in the small quantities $\sin \left( \Delta m^2_{21} / \Delta m^2_{31} \right)$ and $\sin \theta_{13}$:
%One thus obtains, for the channel $\nu_\mu \to \nu_e$ relevant to long-baseline experiments:
%
%\bea
%  P (\nu_\mu \to \nu_e)\!\! & \simeq\!\! & \sin^2 \theta_{23} \sin^2 2 \theta_{13}\, \sin^2 \left( \frac{\Delta m^2_{31} L}{4 E} \right)
%    + \cos^2 \theta_{23} \sin^2 2 \theta_{12}\, \sin^2 \left( \frac{\Delta m^2_{21} L}{4 E} \right)  \nn \\
%  && +\, \frac{1}{2} \cos \theta_{13} \sin 2 \theta_{12} \sin 2 \theta_{13} \sin 2 \theta_{23} \cos \delta
%    \left( \frac{\Delta m^2_{21} L}{4 E} \right) \sin \left( \frac{\Delta m^2_{31} L}{2 E} \right)  \nn \\
%  && -\, \cos \theta_{13} \sin 2 \theta_{12} \sin 2 \theta_{13} \sin 2 \theta_{23} \sin \delta
%    \left( \frac{\Delta m^2_{21} L}{4 E} \right) \sin^2 \left( \frac{\Delta m^2_{31} L}{4 E} \right) .
%\label{eq:numu_nue}
%\eea
%%
%
\bea
  P (\nu_\mu \to \nu_e)\!\! & \simeq\!\! & \sin^2 \theta_{23} \sin^2 2 \theta_{13}\, \sin^2 \left( \frac{\Delta m^2_{31} L}{4 E} \right)
    + \cos^2 \theta_{23} \sin^2 2 \theta_{12}\, \sin^2 \left( \frac{\Delta m^2_{21} L}{4 E} \right)  \nn \\
  && +\, \frac{1}{2} \cos \theta_{13} \sin 2 \theta_{12} \sin 2 \theta_{13} \sin 2 \theta_{23} \cos \delta\,
    \sin \left( \frac{\Delta m^2_{21} L}{4 E} \right) \sin \left( \frac{\Delta m^2_{31} L}{2 E} \right)  \nn \\
  && -\, \cos \theta_{13} \sin 2 \theta_{12} \sin 2 \theta_{13} \sin 2 \theta_{23} \sin \delta\,
    \sin \left( \frac{\Delta m^2_{21} L}{4 E} \right) \sin^2 \left( \frac{\Delta m^2_{31} L}{4 E} \right) .
\label{eq:numu_nue}
\eea
%
The first term corresponds to the dominant, $\Delta m^2_{31}$-driven oscillations;
the second one to the $\Delta m^2_{21}$-driven oscillations; the third and fourth terms involve
both $\Delta m^2_{21}$ and $\Delta m^2_{31}$ and are CP-even and CP-odd, respectively.
Eq.~(\ref{eq:numu_nue}) can be rewritten in the more compact form
%
\be
  P (\nu_\mu \to \nu_e)\, \simeq\, A^2_\oplus + A^2_\odot + 2 \cos \theta_{13} A_\oplus A_\odot
    \cos \left( \frac{\Delta m^2_{31} L}{4 E} + \delta \right) ,
\label{eq:numu_nue_compact}
\eeq
%
where $A_\oplus \equiv \sin \theta_{23} \sin 2 \theta_{13}\, \sin \left( \frac{\Delta m^2_{31} L}{4 E} \right)$
and $A_\odot \equiv \cos \theta_{23} \sin 2 \theta_{12}\, \sin \left( \frac{\Delta m^2_{21} L}{4 E} \right)$.
%
%%
%\bea
%  P (\nu_\mu \to \nu_e)\!\! & \simeq\!\! & \sin^2 \theta_{23} \sin^2 2 \theta_{13}\, \sin^2 \left( \frac{\Delta m^2_{31} L}{4 E} \right)
%    + \cos^2 \theta_{23} \sin^2 2 \theta_{12}\, \sin^2 \left( \frac{\Delta m^2_{21} L}{4 E} \right)  \nn \\
%  & +\!\! & \! \cos \theta_{13} \sin 2 \theta_{12} \sin 2 \theta_{13} \sin 2 \theta_{23}\,
%    \cos \left( \frac{\Delta m^2_{31} L}{4 E} + \delta \right)
%    \sin \left( \frac{\Delta m^2_{21} L}{4 E} \right) \sin \left( \frac{\Delta m^2_{31} L}{2 E} \right) .  \hskip .5cm
%\label{eq:numu_nue_compact}
%\eea
%%
The corresponding formulae for $P (\bar \nu_\mu \to \bar \nu_e)$ can be derived by
switching the sign of the phase $\delta$ in the above expressions, which amounts
in particular to change the sign of the last term in Eq.~(\ref{eq:numu_nue}).
%and of the phase $\delta$
%in the argument of the cosine in the second line of Eq.~(\ref{eq:numu_nue_compact}).
One can quantify the amount of CP violation in the $\nu_\mu$--$\nu_e$ channel
with the CP asymmetry parameter
%
\be
  A_{\mu e}\, \equiv\, \frac{P (\nu_\mu \to \nu_e) - P (\bar \nu_\mu \to \bar \nu_e)}
    {P (\nu_\mu \to \nu_e) + P (\bar \nu_\mu \to \bar \nu_e)}\ \simeq\,
%    \frac{- 8\, J \left( \frac{\Delta m^2_{21} L}{4 E} \right)}{\sin^2 \theta_{23} \sin^2 2 \theta_{13}}\, =\,
    -\, \frac{ \cos \theta_{23} \sin 2 \theta_{12}}{\sin \theta_{23} \sin \theta_{13}}\
    \sin \left( \frac{\Delta m^2_{21} L}{4 E} \right) \sin \delta\, .
\eeq
%


%%%%%%%%%%%%%%%%%%%%%%%%
\subsection{\it Other three-flavour effects    %
\label{subsec:3-flavour}}                              %
%%%%%%%%%%%%%%%%%%%%%%%%

%As discussed in Section~\ref{subsec:vacuum}, long-baseline reactor neutrino experiments
Long-baseline reactor neutrino experiments
such that $\Delta m^2_{21} L / E \sim 1$ are sensitive to $\Delta m^2_{21}$-driven oscillations,
while $\Delta m^2_{31}$- and $\Delta m^2_{32}$-driven oscillations are averaged
due to the limited energy resolution of the detector, leading to the survival
probability~(\ref{eq:nue_nue_2f_improved}). However, an improvement of the energy
resolution could in principle render such experiments sensitive to the subdominant
oscillations governed by $\Delta m^2_{31}$ and $\Delta m^2_{32}$. In this case, the survival
probability of the reactor antineutrinos is given by the three-flavour formula
%
\bea
  P (\bar \nu_e \to \bar \nu_e)\!\! & =\!\! & 1
    - \cos^4 \theta_{13} \sin^2 2 \theta_{12} \sin^2 \left( \frac{\Delta m^2_{21} L}{4 E} \right)
    - \cos^2 \theta_{12} \sin^2 2 \theta_{13} \sin^2 \left( \frac{\Delta m^2_{31} L}{4 E} \right)  \nn \\
  && \phantom{1 } - \sin^2 \theta_{12} \sin^2 2 \theta_{13} \sin^2 \left( \frac{\Delta m^2_{32} L}{4 E} \right) .
\label{eq:nue_nue}
\eea
%
Eq.~(\ref{eq:nue_nue}) describes fast, small-amplitude oscillations governed by
$\Delta m^2_{31}$ and $\Delta m^2_{32}$ developping on top of the dominant
$\Delta m^2_{21}$-driven oscillations. It has been suggested that a precision measurement
of the energy spectrum of reactor antineutrinos at a far detector would make it possible
to distinguish between the normal and inverted mass hierarchies. Indeed, the last two terms
in Eq.~(\ref{eq:nue_nue}) lead to different distorsions of the energy spectrum
depending on whether the hierarchy is normal (in which case $|\Delta m^2_{31}| > |\Delta m^2_{32}|$)
or inverted (in which case $|\Delta m^2_{32}| > |\Delta m^2_{31}|$).



%%%%%%%%%%%%%%%%%%%%%%%%%%
\subsection{\it Neutrino propagation in matter    %
\label{subsec:matter}}                                         %
%%%%%%%%%%%%%%%%%%%%%%%%%%

The interactions of neutrinos with matter (electrons, protons and neutrons) affect
their propagation. This leads to two distinct phenomena: oscillations in matter with
modified parameters with respect to vacuum oscillations, and adiabatic flavour conversion
in a medium of varying density.

Neutrino propagation in matter can be described by a Schr\"odinger-like equation:
%
\be
  i \frac{d}{dt} \left| \nu(t) \right>\, =\, H \left| \nu(t) \right> ,
\label{eq:Schroedinger}
\eeq
%
where $\left| \nu(t) \right>$ is the neutrino state vector at time $t$, and the Hamiltonian $H$
can be split into a free (kinetic energy) part $H_0$ describing neutrino propagation in vacuum,
and a potential term $V$ induced by the interactions of neutrinos in the medium:
%
\be
  H\, =\, H_0 + V\, .
\eeq
%
It is convenient to write the evolution equation~(\ref{eq:Schroedinger}) in
the flavour eigenstate basis $\{\, \left| \nu_e \right>, \left| \nu_\mu \right>, \left| \nu_\tau \right>, \cdots \}$,
where the dots stand for possible additional neutrino species, i.e. sterile neutrinos:
%
\be
  i \frac{d}{dt}\, \nu_\beta(t)\, =\, \sum_\gamma H_{\beta \gamma}\, \nu_\gamma(t)\, , \qquad \qquad
    \beta, \gamma = e, \mu, \tau, \cdots\ .
\label{eq:Schroedinger_components}
\eeq
%
In Eq.~(\ref{eq:Schroedinger_components}),
%where
the $H_{\beta \gamma} \equiv \left< \nu_\beta \right| H \left| \nu_\gamma \right>$ are
the Hamiltonian matrix elements in the flavour basis,
and $\nu_\beta(t) \equiv \left< \nu_\beta | \nu(t) \right>$ %is the component of the neutrino
%state vector along the basis vector $\left| \nu_\beta \right>$,
is the projection of the neutrino state vector onto the basis vector $\left| \nu_\beta \right>$,
%($\left| \nu(t) \right> = \sum_\beta \nu_\beta(t) \left| \nu_\beta \right>$)
i.e. the probability amplitude to find
the neutrino in the flavour eigenstate $\left| \nu_\beta \right>$ at time $t$.
Thus, if the neutrino is produced at $t=0$ in the flavour eigenstate $\left| \nu_\alpha \right>$,
the oscillation probability
%probability of flavour transition
is given by $P(\nu_\alpha \to \nu_\beta; t) = |\nu_\beta(t)|^2$.

Let us first consider the vacuum Hamiltonian. In the flavour basis, it is given by\footnote{This expression
can be easily derived by noting that, in the mass eigenstate basis, the vacuum Hamiltonian is
diagonal with eigenvalues $E_i$:  $\left< \nu_i \right| H \left| \nu_j \right> = E_i \delta_{ij}$.
Using $\left|\nu_\beta \right> = \sum_i U^*_{\beta i} \left|\nu_i \right>$, one then arrives at
$H_{\beta \gamma} = \left< \nu_\beta \right| H \left| \nu_\gamma \right>
= \sum_{i,j} U_{\beta i} U^*_{\gamma j} \left< \nu_i \right| H \left| \nu_j \right>
= \sum_i U_{\beta i} U^*_{\gamma i} E_i$.}:
%\be
%  i \frac{d}{dt}\, \nu_i(t)\, =\, E_i\, \nu_i(t)\, ,
%\eeq
%where $\nu_i (t) \equiv \left< \nu_i | \nu(t) \right>$.
%
\be
  H_0\, =\, U\, \mbox{Diag}\, (E_1, E_2, E_3, \cdots)\, U^\dagger ,  \qquad \qquad  E_i = \sqrt{p^2 + m^2_i}\ ,
\eeq
%
%%
%\be
%  H_0\, =\, U \left(\!\! \begin{array}{ccc} E_1 & 0 & \cdots \\ 0 & E_2 & \cdots \\ \vdots & \vdots & \ddots \end{array}\!\! \right)
%    U^\dagger ,  \qquad \qquad  E_i = \sqrt{p^2 + m^2_i}\ .
%\eeq
%%
%%%  2x2 CaSE   %%%
%\be
%  H_0\, =\, U \left(\!\! \begin{array}{cc} E_1 & 0 \\ 0 & E_2 \end{array}\!\! \right) U^\dagger , \qquad \qquad
%    E_i = \sqrt{p^2 + m^2_i}\ .
%\eeq
%%%%%%%
%
where $p$ is the modulus of the neutrino momentum,
$m_i$ is the mass of the $i^{\rm th}$ neutrino mass eigenstate ($i = 1,2, \cdots n$, with
$n>3$ in the presence of sterile neutrinos) and $U$ is the $n \times n$ lepton mixing matrix.
Assuming ultrarelativistic neutrinos, one can expand
%$E_i = \sqrt{p^2 + m^2_i}\, \simeq\, p +  \frac{m^2_i}{2p}\, \simeq p + \frac{m^2_i}{2E}$
%$E_i = \sqrt{p^2 + m^2_i}\, \simeq\, p + m^2_i/(2p) \simeq p + m^2_i/(2E)$ (identifying $E \simeq p$)
$E_i \simeq p + m^2_i/(2E)$ (in which $E \simeq p$)
and redefine $H_0 \to H_0 - p \mathbf{1}$ to obtain:
%
\be
  H_0\, =\, \frac{1}{2E}\ U\, \mbox{Diag}\, (m^2_1, m^2_2, m^2_3, \cdots)\, U^\dagger\,
    =\, \frac{M^\dagger_\nu M_\nu}{2E}\ ,
\label{eq:H0}
\eeq
%
where $M_\nu$ is the neutrino mass matrix in the flavour basis.
Indeed, removing a piece proportional to the unit matrix from $H$ only affects the overall
%phase of the neutrino state vector, which leaves oscillation probabilities unchanged
phase of the neutrino state vector\footnote{This is true even if this piece is time-dependent.
If $\nu_\beta(t)$ satisfies the evolution equation~(\ref{eq:Schroedinger_components})
with Hamiltonian $H$, then
$\nu'_\beta(t) = e^{i \int_0^t E_0(t')dt'}\, \nu_\beta(t)$ satisfies the same equation  
with the shifted Hamiltonian $H' = H - E_0(t) \mathbf{1}$. $H$ and $H'$ lead to the same
oscillation probabilities since $|\nu'_\beta(t)|^2 = |\nu_\beta(t)|^2$.}, which is unobservable.
%Note that Majorana phases do not affect neutrino oscillations, since the PMNS
%entries appear in the combination $U_{\beta i} U^*_{\gamma i}$ in $H_0$.
In the 2-flavour case, Eq.~(\ref{eq:H0}) reduces to (after subtracting another piece
%$(m^2_1 \sin^2 \theta + m^2_2 \cos^2 \theta)\, \mathbf{1} /(2p)$
proportional to $\mathbf{1}$ from $H_0$):
%
\be
  H_0\, = \left(\!\! \begin{array}{cc}
    - \frac{\Delta m^2}{2E} \cos 2 \theta &  \frac{\Delta m^2}{4E} \sin 2 \theta  \\
    \frac{\Delta m^2}{4E} \sin 2 \theta & 0  \end{array}\!\! \right) ,
\eeq
%
where $\Delta m^2 = m^2_2 - m^2_1$ and $\theta$ is the angle that parametrizes the
$2 \times 2$ lepton mixing matrix.


The matter potential $V$ is induced by coherent forward scatterings of neutrinos
on electrons and nucleons in the medium, which leave the neutrino momentum
unchanged and can therefore interfere with the propagation of the unscattered neutrinos.
It receives a contribution from W boson exchange (charged current) that
%It receives two contributions: one from W boson exchange (charged current), which
is present only for electron neutrinos, as ordinary matter does not contain muons nor taus,
and another one from Z boson exchange (neutral current) that is identical for all neutrino flavours.
%Since lepton flavour is conserved by weak interactions,
The matter potential is diagonal in the flavour basis:
%
\be
  V_{\alpha \beta}\, =\, V_\alpha\, \delta_{\alpha \beta}\,
    = \left( V^{\rm CC}_\alpha + V^{\rm NC}_\alpha \right) \delta_{\alpha \beta}\, ,
\eeq
%
%\be
%  V^{\rm CC}_\alpha\, =\, \begin{cases} \sqrt{2}\, G_F n_e(x) & \alpha = e \\ 0 & \alpha = s \end{cases}\, ,
%    \qquad  V^{\rm NC}_\alpha\, =\, \begin{cases} - \sqrt{2}\, \frac{G_F}{\sqrt{2}}\, n_n(x) & \alpha = e \\ 0 & \alpha = s \end{cases}\, ,
%\eeq
%
where $V^{\rm CC}_\alpha$ and $V^{\rm NC}_\alpha$ depend on the neutrino flavour $\alpha$
($\alpha = e, \mu, \tau$ for active neutrinos, $\alpha = s$ for a sterile neutrino):
%
\be
  V^{\rm CC}_\alpha\, =\, \left\{\!\! \begin{array}{cl}
      \sqrt{2}\, G_F n_e(x) & \quad \alpha = e \\ 0 & \quad \alpha = \mu, \tau, s \end{array} \right. ,  \qquad
  V^{\rm NC}_\alpha\, =\, \left\{\!\! \begin{array}{cl}
%      - G_F n_n(x) / \sqrt{2} & \quad \alpha = e, \mu, \tau \\ 0 & \quad \alpha = s \end{array} \right.\! ,
      - \frac{G_F}{\sqrt{2}}\, n_n(x) & \quad \alpha = e, \mu, \tau \\ 0 & \quad \alpha = s \end{array} \right.\! .
\label{eq:V_CC_NC}
\eeq
%
%%
%\be
%  V^{\rm CC}_\alpha\, =\, \cases{
%      \sqrt{2}\, G_F n_e(x) & $\quad \alpha = e$ \cr 0 & $\quad \alpha = \mu, \tau, s$ } ,  \qquad
%  V^{\rm NC}_\alpha\, =\, \cases{
%%%%      - G_F n_n(x) / \sqrt{2} & $\quad \alpha = e, \mu, \tau$ \cr 0 & $\quad \alpha = s$ } ,
%      - \frac{G_F}{\sqrt{2}}\, n_n(x) & $\quad \alpha = e, \mu, \tau$ \cr 0 & $\quad \alpha = s$ } ,
%\eeq
%%
In Eq.~(\ref{eq:V_CC_NC}), $G_F = 1.166 \times 10^{-5}\, \mbox{GeV}^{-2}$
is the Fermi constant and $n_e(x)$, $n_n(x)$ are the electron and neutron densities
in the medium\footnote{Note that the neutral current contribution to the matter potential
only depends on the neutron density: the proton and electron contributions cancel out
in $V^{\rm NC}$ due to the assumed neutrality of the medium, which implies $n_p = n_e$.},
which a priori depend on the spatial position $x$. For antineutrinos,
the matter potential has the opposite sign:
%
\be
  V_\alpha (\bar \nu)\, =\, - V_\alpha (\nu)\, .
\eeq
%
Summing up, the matter Hamiltonian for neutrinos is given by, in the flavour basis:
%%
%\be
%  H_{\beta \gamma}\, =\, \sum_i U_{\beta i} U^*_{\gamma i} E_i + V_\beta\, \delta_{\beta \gamma}\, .
%\eeq
%%
%
\be
  H_{\beta \gamma}\, =\, \frac{1}{2E}\, \sum_i U_{\beta i} U^*_{\gamma i} m^2_i + V_\beta\, \delta_{\beta \gamma}\, .
\label{eq:H_components}
\eeq
%
%%
%\be
%  H\, =\, \frac{1}{2E}\ U \left(\!\! \begin{array}{ccc} m^2_1 & 0 & \cdots \\ 0 & m^2_2 &  \\ \vdots &  & \ddots \end{array}\!\! \right)
%    U^\dagger + \left(\!\! \begin{array}{ccc} V_e & 0 & \cdots \\ 0 & V_\mu &  \\ \vdots &  & \ddots \end{array}\!\! \right) .
%\eeq
%%
For antineutrinos, the following replacements should be made in Eq.~(\ref{eq:H_components}):
%
\be
  U\, \to\, U^*\, ,  \qquad  V\, \to\, - V\, .
\eeq
%
%as $\left| \nu_\beta \right> = \sum_i U^*_{\beta i} \left| \nu_i \right>$ implies
%$\left| \bar \nu_\beta \right> = \sum_i U_{\beta i} \left| \bar \nu_i \right>$.


%In vacuum, the propagation eigenstates, i.e. the eigenstates of the Hamiltonian,
%coincide with the mass eigenstates $\left|\nu_i \right>$. They are related to the flavour
%eigenstates $\left|\nu_\alpha \right>$ by the lepton mixing matrix $U$, which diagonalizes
%the vacuum Hamiltonian $H_0$:
%%
%\be
%  H_0\, =\, U \left(\!\! \begin{array}{ccc} E_1 & 0 & \cdots \\ 0 & E_2 &  \\ \vdots &  & \ddots \end{array}\!\! \right) U^\dagger\, ,
%%%%%  \qquad  E_i = \sqrt{p^2 + m^2_i}\ ,
%  \qquad \qquad  \left(\!\! \begin{array}{c} \left|\nu_e \right> \\ \left|\nu_\mu \right> \\ \vdots \end{array}\!\! \right)
%    =\, U^* \left(\!\! \begin{array}{c} \left|\nu_1 \right> \\ \left|\nu_2 \right> \\ \vdots \end{array}\!\! \right) ,
%\eeq
%%
%where $E_i = \sqrt{p^2 + m^2_i}\,$.
%Similarly, in matter, the propagation eigenstates $\left|\nu^m_i \right>$ (dubbed {\it matter eigenstates})
%are the eigenstates of the matter Hamiltonian $H$.

In matter, the propagation eigenstates are not the mass eigenstates $\left|\nu_i \right>$ as in vacuum,
but the eigenstates of the matter Hamiltonian $\left|\nu^m_i \right>$, dubbed {\it matter eigenstates}. 
They are related to the flavour eigenstates
by the mixing matrix in matter $U_m$, which diagonalizes $H$:
%
\be
  H\, =\, U_m \left(\!\! \begin{array}{ccc} E^m_1 & 0 & \cdots \\ 0 & E^m_2 &  \\
    \vdots &  & \ddots \end{array}\!\! \right) U^\dagger_m\, ,
  \qquad  \qquad  \left(\!\! \begin{array}{c} \left|\nu_e \right> \\ \left|\nu_\mu \right> \\ \vdots \end{array}\!\! \right)
    =\, U^*_m \left(\!\! \begin{array}{c} \left|\nu^m_1 \right> \\ \left|\nu^m_2 \right> \\ \vdots \end{array}\!\! \right) .
\label{eq:H_diagonalization}
\eeq
%
The eigenvalues $E^m_i$ of the matter Hamiltonian $H$ are called the {\it energy levels in matter}.
%Similarly to $\nu_\beta(t)$, one can define $\nu^m_i(t) = \left< \nu^m_i | \nu(t) \right>$
%as the probability amplitude to find the neutrino in the matter eigenstate $\left| \nu^m_i \right>$
%at the time $t$.
The amplitude of probability to find the neutrino in the matter eigenstate $\left| \nu^m_i \right>$
at the time $t$, $\nu^m_i(t) = \left< \nu^m_i | \nu(t) \right>$, is related to the amplitude
of probability $\nu_\beta(t)$ to find it in the flavour eigenstate $\left| \nu_\beta \right>$
%$\nu_\beta(t) \equiv \left< \nu_\beta | \nu(t) \right>$
by the mixing matrix in matter:
%%
%\be
%  \left(\!\! \begin{array}{c} \nu_e(t) \\ \nu_\mu(t) \\ \vdots \end{array}\!\! \right)
%    =\, U_m \left(\!\! \begin{array}{c} \nu^m_1(t) \\ \nu^m_2(t) \\ \vdots \end{array}\!\! \right) .
%\eeq
%%
%
\be
  \nu_\beta(t)\, =\, \sum_i (U_m)_{\beta i}\, \nu^m_i(t)\, .
\label{eq:nu^m_i-nu_alpha}
\eeq
%
Eq.~(\ref{eq:nu^m_i-nu_alpha}) is nothing but the generalization of the vacuum relation
$\nu_\beta(t)\, =\, \sum_i U_{\beta i}\, \nu_i(t)$.


In many physical contexts, such as neutrino propagation in the Sun, it is a good approximation
to consider the 2-flavour case.
%It is instructive to discuss in greater detail the 2-flavour case, which is a good approximation
%%%To understand the phenomenology of neutrino propagation in matter, it is convenient
%%%to consider the 2-flavour case, which is a good approximation
%to many physical situations, such as the propagation of neutrinos in the Sun.
In the flavour eigenstate basis $\{\, \left| \nu_\alpha \right>, \left| \nu_\beta \right> \}$:
%Assuming that the two flavours are $\alpha = e$ and $\beta = \mu, \tau$ or $s$, one has:
%
\be
  H\, = \left(\!\! \begin{array}{cc}
    - \frac{\Delta m^2}{2E} \cos 2 \theta \pm \sqrt{2}\, G_F n&  \frac{\Delta m^2}{4E} \sin 2 \theta  \\
    \frac{\Delta m^2}{4E} \sin 2 \theta & 0  \end{array}\!\! \right) ,
\eeq
%
with a $+$ sign for neutrinos and a $-$ sign for antineutrinos, and
%%
%\be
%  n\ = \left\{ \begin{array}{ll} n_e(x) & \quad \mbox{if}\ \, \beta = \mu, \tau \\
%    n_e(x) - \frac{1}{2}\, n_n(x) & \quad \mbox{if}\ \, \beta = s \end{array} \right. .
%\eeq
%%
%(if we had considered instead $\alpha = \mu$ or $\tau$ and $\beta = s$, one would
%have $n = - \frac{1}{2}\, n_n(x)$.)
%
\be
  n\ = \left\{ \begin{array}{ll} n_e(x) & \quad \mbox{if}\ \, \alpha = e\ \mbox{and}\ \beta = \mu, \tau \\
    n_e(x) - \frac{1}{2}\, n_n(x) & \quad \mbox{if}\ \, \alpha = e\ \mbox{and}\ \beta = s \\
    - \frac{1}{2}\, n_n(x) & \quad \mbox{if}\ \, \alpha = \mu, \tau\ \mbox{and}\ \beta = s \end{array} \right. .
\eeq
%
The Hamiltonian $H$ is diagonalized by the mixing angle in matter $\theta_m$:
%
\be
  H\, =\, U_m \left(\!\! \begin{array}{cc} E^m_1 & 0 \\ 0 & E^m_2 \end{array}\!\! \right) U^\dagger_m\ , \qquad \qquad
  U_m\, = \left(\!\! \begin{array}{cc} \cos \theta_m & \sin \theta_m \\ - \sin \theta_m & \cos \theta_m \end{array}\!\! \right) ,
\eeq
%
%where (the $-$ sign refers to neutrinos, and the $+$ sign to antineutrinos)
%%
%\bea
%  && E^m_2 - E^m_1\, =\, \frac{\Delta m^2}{2E}\,
%    \sqrt{\left( 1 \mp \frac{n}{n_{\rm res}} \right)^2 \cos^2 2 \theta + \sin^2 2 \theta}\ ,  \\
%  && \sin 2 \theta_m\, =\, \frac{\sin 2 \theta}
%    {\sqrt{\left( 1 \mp \frac{n}{n_{\rm res}} \right)^2 \cos^2 2 \theta + \sin^2 2 \theta}}\ ,  \\
%  && \cos 2 \theta_m\, =\, \frac{\left( 1 \mp \frac{n}{n_{\rm res}} \right) \cos 2 \theta}
%    {\sqrt{\left( 1 \mp \frac{n}{n_{\rm res}} \right)^2 \cos^2 2 \theta + \sin^2 2 \theta}}\ ,
%\eea
%%
where (with a $-$ sign for neutrinos and a $+$ sign for antineutrinos)
%(the corresponding formulae for antineutrinos can be obtained
%by replacing $1 - \frac{n}{n_{\rm res}} \to 1 + \frac{n}{n_{\rm res}}$)
%
\bea
  && E^m_2 - E^m_1\, =\, \frac{\Delta m^2}{2E}\,
    \sqrt{\left( 1 \mp \frac{n}{n_{\rm res}} \right)^2 \cos^2 2 \theta + \sin^2 2 \theta}\ ,  \label{eq:Delta_Em}  \\
  && \sin 2 \theta_m\, =\, \frac{\sin 2 \theta}
    {\sqrt{\left( 1 \mp \frac{n}{n_{\rm res}} \right)^2 \cos^2 2 \theta + \sin^2 2 \theta}}\ ,  \label{eq:sin_theta_m}  \\
  && \cos 2 \theta_m\, =\, \frac{\left( 1 - \frac{n}{n_{\rm res}} \right) \cos 2 \theta}
    {\sqrt{\left( 1 \mp \frac{n}{n_{\rm res}} \right)^2 \cos^2 2 \theta + \sin^2 2 \theta}}\ ,  \label{eq:cos_theta_m}
\eea
%
in which we have introduced the {\it resonance density}
%
\be
  n_{\rm res}\, =\, \frac{\Delta m^2 \cos 2 \theta}{2 \sqrt{2}\, G_F E}\ .
\eeq
%
If $\Delta m^2 \cos 2 \theta > 0$ (resonance condition for neutrinos), the mixing angle
in matter $\theta_m$ is maximal when $n = n_{\rm res}$, irrespective of the (nonzero) value
of the vacuum mixing angle $\theta$:
%
\be
  \sin^2 2 \theta_m\, =\, 1 \quad \mbox{for}\ n = n_{\rm res}  \qquad \qquad
    \mbox{(case $\Delta m^2 \cos 2 \theta > 0$)}\, .
\eeq
%
This the well-known {\it MSW (Mikheev-Smirnov-Wolfenstein) resonance}. For antineutrinos,
the resonance condition is $\Delta m^2 \cos 2 \theta < 0$, and the resonance occurs
for $n = - n_{\rm res}$ (in this case, it is $- n_{\rm res}$ that is positive and can be
interpreted as a resonance density).

%Note that neutrino interactions with matter have
%only a small effect on oscillation parameters when the vacuum mixing angle $\theta$
%is close to maximal ($\sin 2 \theta \simeq 1$), while their effects can be spectacular
%for $\sin 2 \theta \ll 1$.

%Depending on whether the matter density is constant or not, the physics of neutrino
%flavour conversion in matter can be very different. We discuss both cases in turn below.

The physics of neutrino flavour transitions in matter depends on whether the matter density
is constant or not. We discuss both cases in turn below.




%%%%%%%%%%%%%%%%%%%%%%%%%%%%%%%%%%%%%%%%%%
\subsubsection{\it Medium with constant matter density: oscillations in matter  %
\label{subsec:constant}}                                                                                     %
%%%%%%%%%%%%%%%%%%%%%%%%%%%%%%%%%%%%%%%%%%

%Let us first discuss the case where neutrino propagation in a medium with constant matter density,
%$n(x) = \mbox{const.}$. 
Let us first consider the case of a medium with constant matter density, $n(x) = n = const$.
In this case, the Hamiltonian remains constant during the propagation of the neutrinos;
hence the matter eigenstates $\left|\nu^m_i \right>$, the energy levels $E^m_i$ and the mixing
matrix $U_m$ do not depend on time.
%We can therefore rewrite the evolution
%equation~(\ref{eq:Schroedinger_components}) in the basis of matter eigenstates
Inserting Eq.~(\ref{eq:nu^m_i-nu_alpha}) into Eq.~(\ref{eq:Schroedinger_components}), one
then obtains $n$ decoupled evolution equations for the probability amplitudes $\nu^m_i(t)$:
%%
%\be
%   i \frac{d}{dt} \left(\!\! \begin{array}{c} \nu^m_1(t) \\ \nu^m_2(t) \\ \vdots \end{array}\!\! \right)
%    = \left(\!\! \begin{array}{ccc} E^m_1 & 0 & \cdots \\ 0 & E^m_2 & \\ \vdots & & \ddots \end{array}\!\! \right)
%    \left(\!\! \begin{array}{c} \nu^m_1(t) \\ \nu^m_2(t) \\ \vdots \end{array}\!\! \right) ,
%\eeq
%%
%
\be
   i \frac{d}{dt}\, \nu^m_i(t)\, =\, E^m_i \nu^m_i(t)\, ,
\eeq
%
%in which $\nu^m_i(t) = \left< \nu^m_i | \nu(t) \right>$ is the amplitude of probability to find
%the neutrino in the matter eigenstate $\left| \nu^m_i \right>$ at the time $t$, and we have used
%%
%\be
%  \nu_\beta(t)\, =\, \sum_i (U_m)_{\beta i}\, \nu^m_i(t)\, .
%\eeq
%%
which are trivially solved by $\nu^m_i (t) = e^{-i E^m_i t}\, \nu^m_i(0)\, $. Using again
Eq.~(\ref{eq:nu^m_i-nu_alpha}), one arrives at the oscillation probability in matter:
%
\be
  P_m (\nu_\alpha \to \nu_\beta)\, =\, \left| \nu_\beta (t) \right|^2\, =\, \left| \sum_i (U_m)_{\beta i}\, \nu^m_i (t) \right|^2\,
    =\, \left| \sum_i (U_m)_{\beta i} (U_m)^*_{\alpha i}\, e^{-i E^m_i t} \right|^2\, .
%    =\, \sin^2 2 \theta_m \sin^2 \frac{(E^2_m - E^1_m) t}{2}\ .
\label{eq:Pm_alpha_beta}
\eeq
%
%Comparing this expression with the one valid in vacuum, one concludes that neutrino
%oscillations in a medium of constant density are governed by the same formula
Thus neutrino oscillations in a medium of constant density are governed by the same formula
as vacuum oscillations, with the oscillation parameters in vacuum replaced by the
oscillation parameters in matter. More specifically, in the 2-flavour case:
%
\be
  P_m (\nu_\alpha \to \nu_\beta)\, =\, \sin^2 2 \theta_m \sin^2 \frac{(E^2_m - E^1_m) t}{2}
    \qquad \qquad (\alpha \neq \beta)\ .
\eeq
%
%%%As can be seen from Eqs.~(\ref{eq:Delta_Em}--\ref{eq:cos_theta_m}), neutrino interactions
%While neutrino interactions with matter have a rather limited effect on oscillations when the vacuum mixing
%angle $\theta$ is close to maximal ($\sin^2 2 \theta \simeq 1$), their impact can be spectacular
%for $\sin^2 2 \theta \ll 1$ (a condition that is met only by $\theta_{13}$ in practice).
Matter effects can have a spectacular impact on oscillations when the vacuum
mixing angle is small (this is the case only for $\theta_{13}$ in practice).
%(which can be said only from $\theta_{13}$)
%
%Assuming $\sin^2 2 \theta < 1$, one can identify three noticeable regimes:
%{\it (i)} low density ($n \ll |n_{\rm res}|$):
%$\sin^2 2 \theta_m \simeq \sin^2 2 \theta \left( 1 \pm \frac{2 n}{n_{\rm res}}\, \cos^2 2 \theta \right)$,
%(the $+$ sign is for neutrinos, the $-$ sign is for antineutrinos)
%vacuum oscillations dominate, with subleading matter effects;
%{\it (ii)} close to the resonance ($n \simeq |n_{\rm res}|$), oscillations are enhanced
%with respect to vacuum oscillations ($\sin^2 2 \theta_m \simeq 1$) if the resonance condition
%is satisfied, while they are suppressed if it is not satisfied
%($\sin^2 2 \theta_m \simeq \tan^2 2 \theta / (4 + \tan^2 2 \theta)$);
%{\it (iii)} high density ($n \gg |n_{\rm res}|$):
%$\sin^2 2 \theta_m \simeq \tan^2 2 \theta / (\frac{n}{n_{\rm res}})^2$
%oscillations are suppressed by matter effects.
%
Assuming $\sin^2 2 \theta < 1$, one can identify three noticeable regimes:
%
\begin{itemize}
%
\item[{\it (i)}] low density ($n \ll |n_{\rm res}|$):
$\sin^2 2 \theta_m \simeq \sin^2 2 \theta \left( 1 \pm \frac{2 n}{n_{\rm res}}\, \cos^2 2 \theta \right)$,
where the $+$ sign is for neutrinos, and the $-$ sign for antineutrinos.
Vacuum oscillations dominate, with subleading matter effects;
%{\it (ii)} close to the resonance ($n \simeq \pm n_{\rm res}$), if the resonance condition is
%staisfied: $\sin^2 2 \theta_m \simeq 1$ oscillations are enhanced with respect to vacuum oscillations;
%
\item[{\it (ii)}] close to the resonance ($n \simeq |n_{\rm res}|$), oscillations are enhanced
with respect to vacuum oscillations ($\sin^2 2 \theta_m \simeq 1$) if the resonance condition
%$\sin^2 2 \theta_m \simeq 1 - (1 - \frac{n}{n_{\rm res}})^2 / \tan^2 2 \theta$
is satisfied, while they are suppressed if it is not satisfied
($\sin^2 2 \theta_m \simeq \tan^2 2 \theta / (4 + \tan^2 2 \theta)$);
%
\item[{\it (iii)}] high density ($n \gg |n_{\rm res}|$):
$\sin^2 2 \theta_m \simeq \tan^2 2 \theta / (\frac{n}{n_{\rm res}})^2$.
Oscillations are suppressed by matter effects.
%(except if the vacuum angle is close to maximum, i.e. $\sin^2 2 \theta \approx 1$).
%
\end{itemize}

%A remarkable property of matter effects is that they lead to different oscillation probabilities
%for neutrinos and antineutrinos, even in the absence of CP violation. This is a noticeable
%difference with oscillations in vacuum. The matter-induced asymmetry between
%neutrino and antineutrino oscillations is maximal at the resonance.
%This property is used by experiments aiming at determining the mass hierarchy, i.e. whether
%$\Delta m^2_{31} > 0$ or $\Delta m^2_{31} < 0$. In the first case, $n_{\rm res} > 0$
%and neutrino oscillations are enhanced over antineutrino oscillations, while the opposite
%is true in the second case.

%A remarkable difference between oscillations in matter and oscillations in vacuum is that
%the former are different for neutrinos and antineutrinos, even in the absence of CP violation.

%While, in the absence of CP violation, neutrinos and antineutrinos oscillate with the same
%proba\-bility in vacuum, this is not the case in matter.
Due to their different interactions with matter, neutrinos and antineutrinos oscillate with unequal
probabilities when they travel through a medium, even in the absence of CP violation.
This effect %The matter-induced asymmetry
is maximal at the resonance, where depending on the sign of $\Delta m^2 \cos 2 \theta$
either neutrino or antineutrino oscillations are resonantly enhanced.
%but not both.
This property is used by experiments aiming at determining the mass hierarchy, i.e. whether
$\Delta m^2_{31} > 0$ or $\Delta m^2_{31} < 0$. In the first case (normal mass ordering),
%$n_{\rm res} > 0$ and
%$n_{\rm res} = \Delta m^2_{31} \cos 2 \theta_{13} / (2 \sqrt{2} G_F E) > 0$ and
neutrino oscillations are enhanced over antineutrino oscillations,
while the opposite is true in the second case (inverted mass ordering).

Earth matter effects must be taken into account in the analysis of solar and atmospheric
neutrino data\footnote{While the electron density is not constant in the Earth, it is a good
approximation to consider it as made of layers of constant density (the crust, the mantle,
the outer core and the inner core).}.
%hence the above formalism applies in each of these layers.}.
Indeed, at night, neutrinos emitted by the Sun travel through the Earth
before reaching the detector, leading to the so-called {\it Earth regeneration effect}:
part of the solar neutrinos that have oscillated into muon or tau
neutrinos are converted back to electron neutrinos in the Earth. As a result, one can
observe a day-night asymmetry in the solar neutrino data (which is numerically small
in practice, see Section~\ref{subsec:solar}). %\com{modifier cette r\'ef\'erence le cas \'ech\'eant}
The origin of this effect
is simply that oscillation parameters in matter differ from oscillation parameters in vacuum.
In particular, high-energy solar neutrinos exit the Sun in the mass eigenstate
$\left| \nu_2 \right>$ (see Section~\ref{subsec:varying}), which is a propagation eigenstate
in vacuum, but not in matter; %thus they undergo oscillations when they travel through the Earth.
hence they remain in the same state on their way from the Sun to the Earth,
but they oscillate when they travel through the Earth.
Upward-going atmospheric neutrinos are also subject to Earth matter effects,
%as they propagate in the Earth before reaching the detector. This affects
which affects the angular distribution of atmospheric neutrinos. 
The size of the effect is related to the value of the mixing angle $\theta_{13}$
(which controls $\nu_\mu \leftrightarrow \nu_e$ oscillations) and of $\Delta m^2_{21}$,
and also depends on the mass hierarchy and on the precise value of the atmospheric
mixing angle $\theta_{23}$.
%\com{commenter l'importance de ces effets? r\'ef\'erer \`a la section~\ref{subsec:atm}?}


%%%%%%%%%%%%%%%%%%%%%%%%%%%%%%%%%%%%%%%%%%%%%%
\subsubsection{\it Medium with varying matter density: adiabatic flavour conversion    %
\label{subsec:varying}}                                                                                                   %
%%%%%%%%%%%%%%%%%%%%%%%%%%%%%%%%%%%%%%%%%%%%%%

When the density of the medium varies along the neutrino trajectory, $n(x) \neq const$,
the Hamiltonian governing the evolution of the system becomes time-dependent. As a result,
the matter eigenstates, energy levels and mixing angles in matter all depend on time,
and are called {\it instantaneous} quantities. 
The relations~(\ref{eq:H_diagonalization}) and~(\ref{eq:nu^m_i-nu_alpha}) are still valid,
but with instantaneous energy levels $E^m_i(t)$, mixing matrix $U_m(t)$ and matter
eigenstates $\left|\nu^m_i(t) \right>$. It follows that the evolution equations for the probability
amplitudes $\nu^m_i(t) = \left< \nu^m_i(t) | \nu(t) \right>$ are now coupled:
%%
%\be
%  i \frac{d}{dt}\, \nu^m_i(t)\, =\, E^m_i(t) \nu^m_i(t)
%    + i \sum_\gamma\, \frac{d(U^*_m)_{\gamma i}}{dt}\,  (U_m)_{\gamma j}(t)\, \nu^m_j(t)\, ,
%\eeq
%%
%
\be
  i \frac{d}{dt}\, \nu^m_i(t)\, =\, E^m_i(t) \nu^m_i(t)
    - i \sum_\gamma\, (U^*_m)_{\gamma i}(t) (\dot{U}_m)_{\gamma j}(t)\, \nu^m_j(t)\, ,
\eeq
%
where a dot on a quantity means derivative with respect to time, e.g. $\dot{U}_m(t) \equiv \frac{d}{dt}\, U_m(t)$.
In the 2-flavour case, these equations reduce to
%
\be
  i \frac{d}{dt} \left(\!\! \begin{array}{c} \nu^m_1(t) \\ \nu^m_2(t) \end{array}\!\! \right)
    = \left(\!\! \begin{array}{cc} E^m_1(t) & -i \dot{\theta}_m(t) \\ i \dot{\theta}_m(t) & E^m_2(t) \end{array}\!\! \right)
    \left(\!\! \begin{array}{c} \nu^m_1(t) \\ \nu^m_2(t) \end{array}\!\! \right) .
\eeq
%
The terms proportional to $\sum_\gamma\, (U^*_m)_{\gamma i}(t) (\dot{U}_m)_{\gamma j}(t)$
(or $\dot{\theta}_m(t)$ in the 2-flavour case) in the evolution equations induce transitions between
different matter eigenstates $\left|\nu^m_i(t) \right>$ and $\left|\nu^m_j(t) \right>$, which are therefore
no longer propagation eigenstates: a neutrino has a certain probability to ``jump'' from one matter
eigenstate to another while travelling through a medium of varying density.
However, it turns out that in most physical environments (in particular in the Sun), the variation of the
matter density is slow enough that these off-diagonal terms can be neglected.
%these transitions between matter eigenstates occur rarely;
In this case, the evolution of the neutrino system is said to be {\it adiabatic}\footnote{One can define
an adiabaticity criterion by introducing the parameter
$\gamma\ \equiv\ \left. \frac{E^m_2 - E^m_1}{2 |\dot \theta_m(t)|} \right|_{\rm res}\,
    =\ \frac{\Delta m^2 \sin^2 2 \theta}{2 E \cos 2 \theta}\
    \left| \frac{1}{n_{res}} \left( \frac{d n}{d x} \right)_{\rm res} \right|^{-1}$,
where the subscript ``res'' on a quantity means that it should be evaluated in the resonance layer.
The evolution is adiabatic when $\gamma \gg 1$. Given the electron density profile
in the Sun and the values of the oscillation parameters $\Delta m^2_{21}$ and $\theta_{12}$,
this criterion is satisfied for all solar neutrino energies.}: a neutrino produced
in a given instantaneous matter eigenstate will follow the change in matter density during its
propagation and remain in the same matter eigenstate. However,
%the crucial difference with the case of a constant matter density is that the neutrino
its flavour composition will evolve, since the instantaneous matter eigenstates change
as the matter density varies along the neutrino trajectory. This mechanism of neutrino flavour
transition is qualitatively different from oscillations in a medium with constant matter density,
and is called {\it (MSW) adiabatic flavour conversion}.

A particular situation, realized in dense astrophysical environments like supernovae or the Sun,
is the one of {\it level-crossing}. It arises when the electron density in the neutrino production
region (e.g. the center of the Sun) is much larger than the resonant density\footnote{The
terminology ``level-crossing'' is justified by the fact that the asymptotes (corresponding to
the limit $n \to \infty$) of the energy levels cross at the resonance.}. When this
is the case, matter eigenstates approximately coincide with flavour eigenstates.
%For instance, in the two-flavour framework which is well suited to the study of solar neutrinos
%(with $\nu_\alpha = \nu_e$ and $\nu_\beta$ a combination of $\nu_\mu$ and $\nu_\tau$),
%one has $\left|\nu_e\right> \simeq \left|\nu^m_2 (r=0) \right>$ and
%$\left|\nu_\beta\right> \simeq - \left|\nu^m_1 (r=0) \right>$.
For instance, in the center of the Sun, one has $\left|\nu_e\right> \simeq \left|\nu^m_2 (r=0) \right>$
and $\left|\nu_\beta\right> \simeq - \left|\nu^m_1 (r=0) \right>$ (with $\left| \nu_\beta \right>$
a combination of $\left| \nu_\mu \right>$ and $\left| \nu_\tau \right>$) for neutrino energies
$E \gtrsim 5\, \mbox{MeV}$, for which the condition $n_e(r=0) \gg n_{\rm res}$ is satisfied.
Hence high-energy solar neutrinos are produced as quasi pure eigenstates of propagation in matter,
and since their evolution is adiabatic, they remain in the instantaneous matter eigenstate
$\left|\nu^m_2 (r) \right>$ during their propagation. Eventually, they exit the Sun in the mass
eigenstate $\left|\nu_2 \right>$, since by continuity $\left|\nu^m_2 (r) \right> = \left|\nu_2 \right>$
at the radial coordinate $r$ where the electron density vanishes.
%since by continuity $\left|\nu^m_2 (r=R_\odot) \right> = \left|\nu_2 \right>$,
%assuming the electron density vanishes at the Sun radius $R_\odot$.
%
%since by continuity $\left|\nu^m_2 (r=R_\odot) \right> = \left|\nu_2 \right>$,
%where $R_\odot$ is the Sun radius.
In other words, their flavour composition has changed as a result of the evolution
of the instantaneous matter eigenstates: produced as pure electron neutrinos,
they exit the Sun as mass eigenstates, i.e. as admixtures of all neutrino flavours.
Finally, mass eigenstates being eigenstates of propagation in vacuum, they reach
the Earth in the state $\left|\nu_2 \right>$, leading to a survival probability
$P_{ee} = | \langle \nu_e | \nu_2 \rangle |^2 \simeq \sin^2 \theta_{12}$.
Let us add for completeness that low-energy solar neutrinos ($pp$ neutrinos) do not undergo
adiabatic flavour conversions, but rather vacuum oscillations. Indeed, they
are characterized by a higher resonant density for which $n_e(r=0) \ll n_{\rm res}$,
so that matter effects %play a subdominant role.
can be neglected to a good approximation.


%%%%%%%%%%%%%%%%%%%%%%%%%%%%%%%%%%%
\subsubsection{\it Three-flavour oscillations with matter effects   %
\label{subsec:matter_3f}}                                                              %
%%%%%%%%%%%%%%%%%%%%%%%%%%%%%%%%%%%

%%\com{Give and comment the formula by Freund?}




%%%%%%%%%%%%%%%%%%%
\subsection{\it Sterile neutrinos     %
\label{subsec:steriles}}                  %
%%%%%%%%%%%%%%%%%%%

