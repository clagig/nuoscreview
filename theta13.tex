Solar and atmospheric results are well described by two-flavour mixing models, but we know that there are at least three neutrino flavours and therefore at least three mass eigenstates. The experiments described so-far, while giving robust evidence for neutrino oscillation, do not provide a full picture of 3x3 mixing models. 

In order to fully establish the PMNS matrix it is necessary to measure the last mixing angle, \thint. A non-zero value of \thint is required to have CP violation in the lepton sector. The parameter \thint can be measured by reactor neutrino experiments through the measurement of \nueb disappearance at short baselines ($\sim1~km$) or by long-baseline accelerator experiments by looking for electron neutrino appearance in the \num beam.
Reactor experiments directly measure \thint by observing \nueb disappearance according to the simple equation:

\begin{equation}
P(\nueb \rightarrow \nueb) = 1 - \sin^2 2\thint \sin^2(1.267 \dmsqtwo  L/E)
\end{equation}

This is not the case in long-baseline accelerator experiments for which the \nue appearance probability is a sub-leading effect of the oscillation involving \dmsq in which \num mainly oscillate into \nut. The general expression for \papp is a complicated formula that can be derived considering the formalism for the three neutrino families and depends on a combination of \thint, \dcp and matter effects due to the large amount of matter crossed by neutrinos before reaching the detector. An approximated expression of this probability is:

\begin{eqnarray*}
\papp & = & 4C^2_{13}S^2_{13}S^2_{23}\sin^2\phi_{31} \\
& + & 8 C^2_{13}S_{12}S_{13}S_{23}(C_{12}C_{23} \cos \dcp - S_{12}S_{13}S_{23}) \cos\phi_{32}\sin\phi_{31}\sin\phi_{21} \\
& - & 8C^2_{13}C_{12}C_{23}S_{12}S_{13}S_{23} \sin \dcp \sin \phi_{32} \sin \phi_{31} \sin \phi_{21} \\
& + & 4S^2_{12}C^2_{13} (C^2_{12}C^2_{23} + S^2_{12}S^2_{23}S^2_{13} - 2C_{12}C_{23}S_{12}S_{23}S_{13} \cos \dcp ) sin^2\phi_{21}\\
& - & 8C^2_{13} S^2_{13} S^2_{23} \frac{aL}{4E_{\nu}} (1 - 2S^2_{13} ) \cos\phi_{32} \sin \phi_{31} \\
& + & 8C^2_{13}S^2_{13}S^2_{23}\frac{a}{\dmsqtwo}(1-2S^2_{13})\sin^2\phi_{31} \\
\end{eqnarray*}
\begin{equation}
\label{eq:theta13app}
\end{equation}

where $C_{ij} = \cos \theta_{ij}$, $S_{ij} = \sin \theta_{ij}$ and $\phi_{ji}~= \Delta m^2_{ji} L / 4 E_{\nu}$. The terms that include $a$ are a consequence of the matter effects with $a=2\sqrt 2 G_F n_e E_{\nu}~=~7.56\times10^{-5} [eV^2](\rho/(g/cm^3)(E_{\nu}/GeV)$. The term proportional to cos\dcp is invariant for $\nu$ and \nub whilst the term proportional to sin\dcp change if CP is violated. 
The equivalent term for \pappb can be obtained by reversing the signs of the terms proportional to sin\dcp and to $a$. 

These formulas clearly show the complementarity between reactor and long-baseline experiments. The combination of \nueb disappearance from reactors with the measurement of \nue (and eventually \nueb) appearance in long baseline experiments allows to break the degeneracies and access independentely to \thint, \dcp and the sign of $a$.

In this section we will describe the measurements of \nueb disappearance from Daya Bay, RENO and Double Chooz. In addition the combination with measurement of \nueb disappearance provided by The difference between the two channels clearly show the complementarity between reactor and accelerator experiments that can be combined together to measure \dcp and mass hieararchy.
