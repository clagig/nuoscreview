\section{Introduction}
\label{sec:intro}

In the last two decades neutrino physics has undergone a tremendous progress related to the discovery of neutrino oscillations. It is remarkable that after a very long period of theoretical and experimental work, sometimes marked by heated controversies and debates, the original intuition by Bruno Pontecorvo in 1957 \cite{pontecorvo1, pontecorvo2} was at last confirmed in the few years between 1996 and 2003 by the beautiful discoveries of the neutrino oscillations in the atmospheric and in the solar sectors. These discoveries have been properly recognized with the 2015 Nobel Prize in Physics to Takaaki Kajita and Arthur Mc Donald.

Since this milestone, experimental progress has been very fast with the first indications by T2K\cite{t2k2011} of the last mixing angle $\theta_{13}$ and then its discovery by the reactor experiments Double Chooz, RENO and Daya Bay. While experiments with natural sources have been at the forefront of the discovery of neutrino oscillations, recent precision measurements have been mainly been carried out with man-made sources, like nuclear reactors and neutrino beams. 

Today a very large set of experimental results obtained with an amazing variety of experimental configurations and techniques can be interpreted in the framework of the Pontecorvo-Maki-Nakagawa-Sakata (PMNS) framework, consisting of a 3 $\times$ 3 unitary mixing matrix and the mass squared differences of the mass eigenstates. This review covers mainly the rise of this PMNS three neutrinos mixing paradigm and the current status of the experimental determination of its parameters.

As we will discuss later, the next years will continue to see a rich program of experimental endeavour coming to fruition and addressing the three last missing pieces of the puzzle, namely the precision determination of the $\theta_{23}$ value and octant, the unveiling of the neutrino mass ordering and the measurement of the CP-violating phase $\delta$.

The study of neutrinos, and of neutrino oscillations in particular, plays a special role in our understanding of particle physics and at the frontier of our knowledges in this domain. Indeed, the measurement of the neutrino mass as different from zero, whose only experimental evidence comes from neutrino oscillations, is our first positive indication of physics beyond the Standard Model (add reference). Moreover, the global picture of the neutrino mixing matrix is radically different from the analogous CKM mixing matrix in the quark sector and this in itself represents a deep question that needs to be addressed and understood.

Interestingly, the large mixing angles of the PMNS matrix open the possibility for significant CP violation in the lepton sector. These studies might help our understanding of the baryon asymmetry in the universe, whose best interpretive framework is today represented by the leptogenesis model.
 
This review is organized as follows. We first introduce the massive neutrino mixing formalism, with particular emphasis on a pedagogical approach for students and scholars entering this field. Then we discuss the experimental evidence for neutrino oscillations in the three historical sectors, namely the study of solar neutrinos and $\theta_{12}$ angle, then the atmospheric neutrinos and the $\theta_{23}$ angle and finally the last discovered sector parametrized by the $\theta_{13}$ angle. In each of these sectors, the relation between experimental results and theory can be easily understood on the basis of effective two neutrino mixing formulae, even though the experimental precision increasingly requires to take into account higher order effects. We then turn to the new phenomena that can only be interpreted as genuine three neutrino effects, and especially the study of CP violation effects that has been initiated by the T2K and NOvA experiments. To conclude this section, we briefly summarize our current level of understanding of the various parameters of the PMNS framework and the role played by global fits.

After briefly mentioning the remaining anomalies that can not be interpreted inside the PMNS framework, we turn to the presenting an overview of the future experimental program and how it can answer the remaining questions. 
