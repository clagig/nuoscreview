\section{ PMNS model: summary}
\label{sec:summary}


The large set of experimental results on neutrino oscillations, with the exception of the anomalies that will be presented in a later section, supports the global picture of three active neutrino mixing parametrized by the PMNS mixing matrix.

Global fits have been performed on the neutrino oscillation data by two groups~\cite{nufit,fogli}.
The results of the most recent fit~\cite{nufit} are reported in Table~\ref{tab:globalfit}.


\begin{table}
\centering
\begin{tabular}{|c|c|c|}
  \hline
  Parameter & Normal Ordering & Inverted Ordering  \\ 
  \hline
$\theta_{12}$ (deg)& $33.56^{+0.77}_{-0.75}$ &  $33.56^{+0.77}_{-0.75}$\\  
  $\theta_{23}$ (deg)& $41.6^{+1.5}_{-1.2}$ &  $50.0^{+1.1}_{-1.4}$\\  
  $\theta_{13}$ (deg)& $8.46^{+0.15}_{-0.15}$ & $8.49^{+0.15}_{-0.15}$ \\  
  $\delta_{CP}$ (deg)&  $261^{+51}_{-59}$& $277^{+40}_{-46}$ \\  
  $\Delta m²_{21}$ ($10^{-5}$eV$²$)& $7.50^{+0.19}_{-0.17}$ & $7.50^{+0.19}_{-0.17}$ \\  
  $\Delta m²_{3l}$ ($10^{-3}$eV$²$)&  $2.524^{+0.039}_{-0.040}$&  -$2.514^{+0.038}_{-0.041}$\\  
  \hline
\end{tabular}
\caption{
PMNS parameters determined by a recent global fit to the world neutrino data \cite{nufit} in the hypothesis of normal ordering (second column) and inverted ordering (third column). The parameter $\Delta m²_{3l}$ is equal to $\Delta m²_{31}$ for NO and to -$\Delta m²_{32}$ for IO. }
\label{tab:globalfit}
\end{table}


At the moment, there is no significant preference for the normal or inverted ordering of the neutrino mass eigenstates. The measurement of the angles $\theta_{12}$ and 
$\theta_{13}$ and the mass squared differences $\Delta m² $ has already reached the percent precision level. This is not the case for the angle $\theta_{23}$ where the three $\sigma$ range spans the interval (38,52) degrees (see also Fig.xx) because of a mirror solution in the higher octant. Indeed, even the non-maximality of $\theta_{23}$ is not firmly established. It must be noticed that the fit reported in Table~\ref{globalfit} does not use the most recent Super-Kamiokande atmospheric results. 

The precise determination of $\theta_{23}$, of the CP-violating phase $\delta$ and of the mass ordering remains a task for future experiments.

Another long standing feature of the underlying data revealed by these fits is the 2 $\sigma$ tension between the determination of $\Delta m²_{21}$ by KamLAND on one side, and using the solar neutrino results by Super-Kamiokande, SNO and Borexino on the other side.
This tension is related to the non observation of the turn up on the lower part of the energy spectrum as predicted by the MSW effect. The observation of the day-night effect for solar neutrinos by Super-Kamiokande is also contributing to this tension. (expliquer)  

