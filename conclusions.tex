\section{Conclusions}
\label{sec:conc}

The field of neutrino oscillations has recently shown a rapid progress and is now entering the precision period. Indeed, after the establishment of neutrino oscillations as the mechanism behind the anomalies observed in the study of solar and atmospheric neutrinos, the measurement of the last mixing angle \thint, several confirmations and more accurate measurements have established the PMNS three active neutrino paradigm as the framework capable of interpreting a vast array of experimental results. 

A few anomalies, most notably the LSND anomaly, can not be interpreted in the PMNS framework. Recent experimental results have not confirmed these anomalies, that have been the object of an intense scrutiny and debate. An experimental program is underway that will provide new more sensitive tests of these anomalies, with reactor neutrinos, a radioactive source and with short baseline accelerator based experiments. These experiments will provide their results in the next few years, before 2020. Therefore in a few years we should be able to either remove these anomalies from the list of open questions, or on the contrary face a major exciting discovery.  

For most of the parameters of the PMNS matrix and for the squared mass differences the measurements have attained the \% precision and certainly great attention needs to be devoted to the experimental systematic uncertainty to make further progress. A robust and precise control of the neutrino flux is needed for future experiments devoted to the improvement of the experimental accuracy. This calls for a full fledged program of auxiliary experiments and measurements, for instance in the field of hadro-production experiments like NA61/SHINE. It calls also for a renewed program to establish a reliable model for the neutrino nucleus cross-section, where new phenomenological investigations and new high precision data will be major ingredients. The goal is to reach a control at the \% level, as opposed to 10\% or worse uncertainty level of the current uncertainties.

A rich experimental program is today under preparation to answer the remaining open questions, like the precise measurement of the $\theta_{23}$ mixing angle, its deviation from the maximal value $\pi/4$ and its octant, the question of the mass ordering and the determination of the CP violating phase \dcp. These are fundamental questions and measurements in this field are likely to dominate the scene of experimental particle physics in the next years. Partial answers to these questions will come from the running long baseline experiments T2K and \nova.
The question of mass ordering will be addressed to a certain extent by non-accelerator program using either atmospheric neutrinos or nuclear reactors. 
In the next decade two major projects, DUNE and Hyper-Kamiokande, are in preparation to provide high precision measurements. 






