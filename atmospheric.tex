\section{The 2-3 atmospheric sector }

In this subsection we will describe the status of the so called atmospheric neutrino oscillations, i.e. the 2-3 oscillations that were first discovered in the study of atmospheric neutrinos and were later precisely measured with the use of neutrino beams produced by particle accelerators.   


\subsection{The atmospheric neutrino flux}

The atmosphere is constantly bombarded by primary cosmic rays, composed mainly of protons, with a smaller component, $\simeq$ 5\%, of $\alpha$ particles, and an even smaller fraction of heavier nuclei. The interaction of these particles with atomic nuclei in the atmosphere produces hadronic showers composed mainly of pions and kaons. The decays of these mesons according to 
$\pi^+ \rightarrow \mu^+ \nu_\mu$ followed by 
$\mu^+ \rightarrow e^+ \nu_e \bar{\nu}_\mu $,
$K^+ \rightarrow \pi^+ \nu_\mu$ and $K_L \rightarrow \pi^+ e^- \nu_e$,
together with their charge conjugated processes, produces a flux of $\nu_\mu$ and $\nu_e$ with a steeply falling power-law spectrum (Fig. \ref{fig:nuatmflux}), approximately $E^{-2.7}$ in the region above 1 GeV.   


\begin{figure}[htbp]
\begin{minipage}[c]{.46\linewidth}
%\begin{minipage}[c]
   	      \includegraphics[width=0.9\linewidth]{figures/fig7a-c.pdf}
   \end{minipage} \hfill
   \begin{minipage}{.46\linewidth}
      \includegraphics[width=0.9\linewidth]{figures/fig7b-c.pdf}
   \end{minipage}
    \caption{The predicted flux of atmospheric neutrinos (left) as computed by several groups and the flux ratios (right)~\cite{PhysRevD.83.123001}. Courtesy of M. Honda et al. }
 \label{fig:nuatmflux}
\end{figure}


The calculation of this neutrino flux~\cite{Gaisser:2002jj} relies on the knowledge of the primary cosmic ray flux and composition, of the Earth magnetic field, and of the hadro-production cross-section on the light nuclei of the Earth atmosphere. Recent studies~\cite{PhysRevD.83.123001,Barr:2004br,Battistoni:2002ew,honda2015} taking into account three-dimensional effects and recent measurements of the hadro-production cross-sections largely improve on previous efforts and reach precisions of 7-8 \% for the flux in the 1-10 GeV range. It should be noticed that the ratio $N(\nu_\mu + \bar{\nu}_\mu)/N(\nu_e + \bar{\nu}_e)$ is predicted with a much better precision of a few \% as several systematic uncertainties cancel out in this ratio. In the limiting case where all the muons from pion decays decay in flight, this ratio is close to 2, as can be easily deduced from the decay processes mentioned above.

%\subsection{The early days, the up-down asymmetry and controversy}

\subsection{Interaction of high energy neutrinos with nuclei}

The interaction of neutrinos with matter takes place through charged current (CC) interactions (with the exchange of a $W$ boson and the production of a charged lepton in the final state) or neutral current (NC) interactions (with the exchange of a $Z^0$ boson). 
Above a few hundred MeV neutrino energy interactions with the nucleus have a typical $q^2$ which corresponds to a scale of the order of one fermi, and the neutrino scatters off individual nucleons inside the nucleus. These interactions (Fig.~\ref{fig:xsec}) can be classified as:
\begin{itemize}
  \item (Quasi-) Elastic interactions, where the final state nucleon is ejected from the nucleus as a proton or neutron, like $\nu_\mu \: n \rightarrow \mu \: p$
  \item resonant single pion production like 
  $\nu_\mu \: n \rightarrow \mu \: N^*$ followed by $N^* \rightarrow n \pi$ 
  where $N^*$ can be  a resonance like the $\Delta$,
  \item Deep Inelastic scattering at large energies and for large momentum transfers, where the neutrino interacts with a quark inside the nucleon.
  \end{itemize}  

The Charged Current Quasi Elastic (CCQE) process $\nu_l \: n \rightarrow l \: p$ where $l=e,\mu$ plays a special role because it is the dominant process below a neutrino energy of 1 GeV. Moreover its simple final state is very suitable to the reconstruction of the neutrino energy based on the measurement of the outgoing charged lepton. Neglecting Fermi momentum, the reconstructed neutrino energy is 
\begin{equation}
E_\nu = \frac{m_p^2 - (m_n-E_b)^2 - m^2_l + 2 (m_n-E_b)E_l}{2 (m_n - E_b - E_l + p_l \cos \theta)}
\end{equation}
 where $E_b$ is the effective binding energy necessary to extract the nucleon from the nucleus, $m_p$, $m_n$ and $m_l$  are the proton, neutron and lepton mass, $E_l$, $p_l$ and $\theta$ are the measured lepton energy, momentum and angle with respect to the incoming neutrino direction. 
 In Cherenkov detectors, where the protons and most of the pions in these interactions are below the detection threshold, this process is particularly useful because the electrons and the muons can be easily detected, reconstructed and identified. 

In the simplest approach to the calculation of the cross-section, the impulse approximation, that is the scattering from individual nucleons, is used and the nucleus is simply parametrized by the Fermi momentum and the binding energy $E_b$. The CCQE cross-section is then mainly governed by the axial vector form factor of the nucleon, where usually a dipole form is assumed
$ G_A (q²) = \frac {g_A} {(1+q^2/M^2_A)^2}$.
Most parameters are determined from electron scattering for the vector form factors, and from nuclear $\beta$ decays for $g_A$, leaving the axial mass value $M_A$ as the main free parameter. This simple approach failed, as revealed by the fact that the value $M_A =  1.35 \pm 0.17$ GeV/c$^2$ measured MiniBoone~\cite{miniboone-ccqe} for interactions on carbon was very different from the value 1.03 GeV/c$^2$ obtained from the analysis of neutrino interactions on deuterium (Fig.~\ref{fig:xsec}).

\begin{figure}[htbp]
\begin{minipage}[c]{.46\linewidth}
%\begin{minipage}[c]
   	      \includegraphics[width=0.9\linewidth]{figures/cc_inclusive_nu.pdf}
   \end{minipage} \hfill
   \begin{minipage}{.46\linewidth}
 %     \includegraphics[width=0.9\linewidth]{figures/CCQE_C_2.pdf}
            \includegraphics[width=0.9\linewidth]{figures/CCQE_C_3.pdf}
   \end{minipage}
    \caption{ Left plot: Total CC neutrino per nucleon cross sections (for an isoscalar target) divided by neutrino energy and
plotted as a function of energy~\cite{formaggio}. 
Right plot: CCQE cross-section on carbon~\cite{alvarez}.
The solid
lines denote various theoretical models.  The dash-dotted and
dotted lines are two models with $M_A=1$ and 1.35 GeV respectively. 
The dashed line takes into account the 2p-2h processes.  The data points
are from MiniBooNE. Courtesy of J. Formaggio et al. and L. Alvarez-Ruso.}
 \label{fig:xsec}
\end{figure}

A partial solution to this discrepancy came from considering additional processes like interactions of the neutrino with a correlated pair of nucleons inside the nucleus. These processes are known to exist in electron-nucleus scattering. From the experimental point of view, these processes, named two particles-two holes excitations (2p-2h), cannot be distinguished from CCQE interactions, as low momentum protons and neutrons are generally not detected. These processes can represent 10-20 \% of the total CCQE cross-section. Despite this progress, the normalization and shape of this additional contribution is still not well known and model dependent.

Other nuclear effects, like Final State Interactions (where the ejected nucleon reinteracts within the same nucleus), or Secondary Interactions (where the nucleon interacts with other nuclei in the detector) introduce smearing and corrections which are under evaluation, especially for relatively large nuclei like Carbon, Oxygen, Argon and Iron generally used in these experiments. 
The status of the anti-neutrino cross-sections is even less established, as is the case for the $\nu_e$ cross-sections required by the present and future long-baseline oscillation experiments.  

All these cross-sections are the object of an intense theoretical and phenomenological activity~\cite{zeller,martini} given their relevance for oscillation analyses. The knowledge of these cross-sections is still model-dependent and the precision does not exceed the 10-20 \% level. Therefore, most of the accelerator experiments include near detectors to study the event rates before oscillations intervene. Clearly a more focused effort, combining phenomenological developments and well-controlled experimental data is required for future high-precision oscillation experiments.


\subsection{The evidence for atmospheric neutrino disappearance}
\label{subsec:atmevidence}
%from Super-Kamiokande}

The study of atmospheric neutrinos started in the 1960s: two experiments in very deep mines, in South Africa~\cite{Reines:1965qk} and in India~\cite{Achar:1965ova}, observed muons produced by atmospheric neutrino interactions. 
In the 1980s, several massive underground experiments, mainly motivated by the search for the proton decay predicted by Grand Unification theories, started collecting data. These experiments needed to study in detail atmospheric neutrino interactions as they constitute a background for proton decay searches.

In 1988, Kamiokande~\cite{kam88} reported a deficit in the number of $\nu_\mu$ single-ring candidates (85 events observed versus 144 expected), while the number of $\nu_e$ single-ring candidates agreed with the prediction (93 events observed versus 88.5 expected). 
This initiated the so-called atmospheric neutrino anomaly. A similar deficit was also observed by the IMB experiment and later by MACRO and SOUDAN-2, while the Fr\'ejus and NUSEX experiments observed no deficit. 

The situation evolved rapidly with the advent of Super-Kamiokande~\cite{sknim}, that started data-taking in 1996. Super-Kamiokande is a very large water Cherenkov detector located in the Mozumi mine (Gifu prefecture, Japan), under a 1000 m rock overburden, equivalent to 2700 m of water. It is a stainless steel tank (41.4 m high, 39.3 m diameter) containing 50 kt of ultra-pure water. The detection volume is partitioned in an outer detector, composed of 1885 8-inch PMTs, and an inner detector with 11146 20-inch PMTs. The fiducial volume is 22.5 kt. Super-Kamiokande could rapidly accumulate a rather large data set of atmospheric neutrinos, measuring the direction of the produced lepton, its energy for fully contained events and their nature. Above a neutrino energy of a few hundred MeV, the direction of the produced lepton is strongly correlated with the direction of the incoming neutrino.

Super-Kamiokande classifies events as Fully Contained (where the neutrino interaction takes place inside the inner detector and all the particles stop in the same volume), Partially Contained (some particles escape to the outer detector) and Upward Going muons, where the neutrino interacts in the rock below the detector. The events can be further classified according to the number of Cherenkov rings observed, the observed total energy, the number of decay electrons. Electrons and muons can be reliably identified (the misidentification probability is 0.7 \%) on the basis of the ring properties. Cherenkov rings produced by muons have sharp edges while in the case of electrons the photons are produced by the numerous electrons and positrons in the electromagnetic shower, with angular dispersion due to scattering, resulting in a ring with diffuse edges. The momentum resolution for an isolated ring produced by a lepton of momentum $p$ is $0.6\% +2.6\%/\sqrt{p ({\rm GeV/c})}$. Sub-threshold muons and charged pions can be tagged by the presence of a delayed electron from muon decay, called Michel electrons.

In 1998, the Super-Kamiokande collaboration presented their first analysis of atmospheric neutrinos~\cite{Fukuda:1998mi}, in particular the distributions of zenith angle for $ \nu_\mu$ and $\nu_e$ event selections (see Fig.~\ref{fig:sk-atm} for an updated distribution) based on an exposure of 33 kton year. This was the first compelling evidence for neutrino oscillations as the explanation of the previously mentioned anomaly.  

Indeed the neutrino path from the production to the detection varies from 15 km for down-going neutrinos (cosine of the zenith angle equal to 1) to more than 12000 km for up-going neutrinos having traversed the whole Earth (cosine of the zenith angle equal to -1), thereby probing a large span of possible oscillation lengths. 

In a two neutrino scenario, the $\nu_\mu$ disappearance is governed by (cf. Eq.~(\ref{eq:numudisappApp})) 
\begin{equation}
P(\nu_\mu \rightarrow \nu_\mu) \simeq 1 - \sin^2 2 \theta_{23} \sin^2 (\frac{\Delta m^2_{31} L}{4 E})\label{eq:mudisapp}.
\end{equation}
%where $\theta$ is the relevant mixing angle and $\Delta m^2_{atm}$ the effective squared-mass difference of the mass eigenstates. 
A glance at Fig.\ref{fig:sk-atm} reveals several important overall features: 
\begin{itemize}
\item there is a strong disappearance of $\nu_\mu$, especially visible for up-going neutrinos. As the survival probability for very long baseline approaches $1- 1/2 \sin^2 2 \theta_{23}$, and the observed survival probability is close to 0.5, the mixing angle is therefore close to the maximal value $\pi/4$. 
\item The disappearance sets in for neutrinos close to horizontal zenith angle, and therefore the oscillation length should be of the order of 400 km for an energy around 1 GeV, or $\Delta m^2_{31} \simeq 10^{-3}$eV$^2$.  
\item There is no sizeable excess or deficit of $\nu_e$. Therefore the oscillations of $\nu_{\mu}$ should mainly involve either $\nu_{\mu} \rightarrow \nu_{\tau}$ or $\nu_{\mu} \rightarrow \nu_s$, where $\nu_s$ is an additional neutrino state.
\end{itemize}


\begin{figure}[htbp]
\centering
\includegraphics[width=0.8\linewidth]{figures/sk-2006-atm.pdf}
  \caption{Zenith angle distributions of Super-Kamiokande atmospheric neutrino events from~\cite{Hosaka:2006zd}. Fully contained
e-like, $\mu$-like events
are shown for data (filled circles with statistical
error bars), MC distributions without oscillation (boxes)
and
best-fit distributions (dashed). The box height shows the
statistical error. Courtesy of the Super-Kamiokande collaboration.}
 \label{fig:sk-atm}
 \end{figure}

Independently of any accurate prediction of the neutrino flux, the experimental observation of the distributions of Fig.\ref{fig:sk-atm} is sufficient to make a strong case for neutrino disappearance. Indeed, above a few GeV, the neutrino flux is isotropic, as the primary cosmic rays are not deflected in a significant way by the geomagnetic field. The observation of a zenith angle dependent deficit is therefore a sufficient argument to conclude that the flux of the different neutrino species is not conserved.  

While in 1998 other hypotheses like decay or decoherence were still open, more recent data from long baseline accelerator experiments together with the study of the L/E distribution by Super-Kamiokande have ruled out all explanations apart from oscillations because the alternative hypotheses imply a different L/E behaviour. 
    
The IceCube experiment at the South Pole has recently completed the installation of DeepCore, a denser array of optical modules, aimed at significantly lowering the muon threshold. With data recorded between 2011 and 2014, corresponding to 5074 observed events, they have recently published an analysis~\cite{Aartsen2016161} of the disappearance of atmospheric $\nu_\mu$   in the range 10-100 GeV, requiring the zenithal angle to satisfy $\cos \theta < 0$,  which has a similar sensitivity to that of Super-Kamiokande (Fig. \ref{fig:icecubeosc}) with the prospect of further improvements. 


\begin{figure}[htbp]
\centering
%\includegraphics[width=0.5\linewidth]{energy_miniboone.eps}
\includegraphics[width=0.6\linewidth]{figures/icecube_osc2014_data_mc_LE.pdf}
  \caption{Distribution of atmospheric neutrino events measured by the IceCube experiment~\cite{Aartsen2016161} as a function of the
reconstructed L/E. Data are compared to the best fit and
expectation with no oscillations (top), and the ratio of data
and best fit to the expectation without oscillations is also shown
(bottom). Bands indicate estimated systematic uncertainties. Courtesy of the IceCube collaboration.}
 \label{fig:icecubeosc}
 \end{figure}
